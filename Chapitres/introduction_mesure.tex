%\documentclass[a4paper,11pt,answers]{article}
%
%\usepackage{paf}
%
%\title{Introduction à la théorie de la mesure}
%\date{2017}
%
%\begin{document}
%\maketitle

\section{Cardinaux, sommes dénombrables, nouveau formalisme sur les ensembles}

\subsection{Cardinalité}

On note $\abs{E}$ le cardinal de $E$.

Deux ensembles $E$ et$F$ sont de même cardinal s'il existe une bijection entre eux.

Par ailleurs, on a $\abs{E} \leq \abs{F}$ lorsqu'il existe une injection de $E$ dans $F$. C'est bien une relation d'ordre sur les cardinaux.

On définit de plus $\abs{E}<\abs{F}$ lorsque $\abs{E} \neq \abs{F}$ et $\abs{E} \leq \abs{F}$.

Un ensemble $E$ est dit dénombrable lorsque $\abs{E} = \abs{\N}$.

On dispose du résultat fondamental suivant:
\begin{prop}[Cardinalité de l'ensemble des parties de $E$]
On a $\abs{\mathcal{P}(E)} > \abs{E}$.
\end{prop}

\begin{proof}
C'est un argument de Cantor.

Il existe une injection de $E$ dans $\mathcal{P}(E)$: l'injection canonique

$x \mapsto \left\{ x \right\}$.

Supposons maintenant qu'il existe une injection de $\mathcal{P}(E)$ dans $E$. 

Notons $f: \mathcal{P}(E) \to E$ cette injection.

Considérons $X=\left\{ x \in E / \, x \notin f^{-1}(x) \right\}$. On peut ici parler de $f^{-1}(x)$ en tant que partie de $E$ car $f$ est injective.

Dans ce cas, que dire de $y=f(X)$? Si $y \in X$, on aboutit à une absurdité car $y \in f^{-1}(y)$ donc $y \notin X$. De même, si $y \notin X$, cela entraîne $y \in X$. $f$ ne peut donc pas exister.
\end{proof}

Les ensembles $\Z$, $\Q$ sont dénombrables. De manière générale, pour tout entier $p$ et pour tout ensemble $E$ dénombrable, $E^p$ est dénombrable. Il suffit pour cela d'utiliser un ordre de type alphabétique sur les p-uplets.

En utilisant la fonction bijective $\tanh$, on peut montrer que $]-1;~1[$ est de même cardinal que $\R$.

De même, en utilisant les écritures binaires des nombres de $]-1;~1[$, on peut montrer que $]-1;~1[$ est de même cardinal que $\left \{ 0,~1\right\}^{\N}$.

Et enfin, toute partie de $\N$ pouvant être désignée par un unique élément de $\left \{ 0,~1\right\}^{\N}$, on a la série d'égalités:

\[
\abs{\mathcal{P}\left(\N\right)} = \abs{\left \{ 0,~1\right\}^{\N}} = \abs{]-1;~1[} = \abs{\R}
\]

De même, si les $\left(E_n\right)_{n \in \N}$ sont tous dénombrables alors $\bigcup \limits_{n \in \N} E_n$ l'est aussi. Il suffit pour cela de noter qu'un élément de $\bigcup \limits_{n \in \N} E_n$ peut être désigné par un couple d'entiers.

\subsection{Sommes de nombres positifs}

\begin{prop}[Somme bornée de nombres positifs]
Si $\displaystyle{\sum \limits_{i \in I}} x_i < +\infty$ et si, pour tout $i \in I$, $x_i \geq 0$ alors l'ensemble $K = \left\{ i / x_i>0\right\}$ est au plus dénombrable.
\end{prop}

\begin{proof}
Notons que $K \bigcup \limits_{n \in \N^{*}} K_n$ avec $K_n = \left\{ i / x_i > \dfrac{1}{n}\right\}$.

On sait d'autre part que 

\[
\displaystyle{\sum \limits_{i \in I}} x_i \geq \displaystyle{\sum \limits_{i \in K_n}} x_i \geq \dfrac{\abs{K_n}}{n}
\]

En particulier, pour tout $n$, $\abs{K_n} \leq n \displaystyle{\sum \limits_{i \in I}} x_i < +\infty$. Ainsi, les $K_n$ sont tous de cardinaux finis. Cela permet de conclure!
\end{proof}

\begin{prop}[Permutations des termes d'une somme dénombrable de réels positifs]

On considère la somme $\displaystyle{\sum \limits_{n \in \N}} x_n$ où les $x_n$ sont tous positifs ou nuls.

Soit une permutation $\sigma$ de $\displaystyle{\sigma}\left(\N\right)$.

Alors 
\[\displaystyle{\sum \limits_{n \in \N}} x_n = \displaystyle{\sum \limits_{n \in \N}} x_{\sigma(n)}\]
\end{prop}

\begin{proof}
On pose $S=\displaystyle{\sum \limits_{n \in \N}} x_n$

On va commencer par supposer $S=+\infty$.

Soit $M \in R$. Il existe $N$ tel que $\displaystyle{\sum \limits_{0 \leq n \leq N}} x_n > M$. 

Il nous faut prouver qu'il existe aussi $N'$ tel que $\displaystyle{\sum \limits_{0 \leq n \leq N'}} x_{\sigma(n)} > M$.

Considérons pour cela l'ensemble $E_n = \left\{\sigma^{-1}(n)/ \, 1 \leq n \leq N\right\}$. Cet ensemble admet un maximum noté $N'$. En particulier, $\intint{0}{N'} \supset E_n$ et donc $\sigma\left(\intint{0}{N'}\right) \supset \intint{0}{N}$. On a donc:

$\displaystyle{\sum \limits_{0 \leq n \leq N'}} x_{\sigma(n)} \geq \displaystyle{\sum \limits_{0 \leq n \leq N}} x_n > M$.

Ce qui prouve bien que $\displaystyle{\sum \limits_{n \in \N}} x_{\sigma(n)} = +\infty$.

Le cas où $S<+\infty$ se traite de la même manière.
\end{proof}

\begin{prop}[Réarrangement de termes dans une somme dénombrable]
Soit $\left(I_i\right)_{i \in \N}$ une partition dénombrable de $\N$ et une somme $\displaystyle{\sum \limits_{n \in \N}} x_n$ où les $x_n$ sont tous positifs ou nuls.

Alors,
\[
\displaystyle{\sum \limits_{n \in \N}} x_n = \displaystyle{\sum \limits_{i \in \N}} \sum \limits_{k \in I_i} x_k
\]
\end{prop}

\begin{proof}
Cela se prouve selon le même principe que précédemment. Pour changer, on va considérer que $\displaystyle{\sum \limits_{n \in \N}} x_n = S < +\infty$.

Pour tout $n$ et $p$, on a, $\displaystyle{\sum \limits_{i \in \intint{0}{p}}} \sum \limits_{k \in I_i \cap \intint{0}{n}} x_k \leq S$.

Ainsi, par passage à la limite sur $n$ puis sur $p$, $\displaystyle{\sum \limits_{i \in \N}} \sum \limits_{k \in I_i} x_k \leq S$.

De plus, pour tout $\varepsilon > 0$, il existe un rang $n$ tel que $S-\varepsilon \leq \displaystyle{\sum \limits_{k \in \intint{0}{n}}} x_k$. 

Mais dans ce cas, il existe un rang $p$ tel que $\intint{0}{n} \subset \bigcup \limits_{i \in \intint{0}{p}} I_i$. Et ainsi, pour tout $m \geq p$, $S-\varepsilon \leq \displaystyle{\sum \limits_{i \in \intint{0}{m}}} \sum \limits_{k \in I_i} x_k \leq S$.
\end{proof}

\begin{prop}[Théorème de convergence absolue pour les séries]
Soit une série $\displaystyle{\sum \limits_{n \in N}} x_n$ de nombres quelconques.

Si $\displaystyle{\sum \limits_{n \in N}} \abs{x_n} < \infty$ alors cette série converge.
\end{prop}

\begin{proof}
Cela se prouve en considérant $S_n = S_n^{+}+S_n^{-}$ où les $S_n^{+}$ et les $S_n^{-}$ sont les sous-séries de termes positifs (resp. négatifs).
\end{proof}

\subsection{Limites d'ensembles, nouveau formalisme}

Soit $\left(E_n\right)_{n \in \N}$ une suite d'ensembles tous inclus dans un ensemble $E$.

On pose $\sup E_n = \bigcup \limits_{k \geq n} E_k$ et $\inf E_n = \bigcap \limits_{k \geq n} E_k$.

La suite des $\sup E_n$ est une suite décroissante en ce sens que, pour tout $n$, $\sup E_n \supset \sup E_{n+1}$.

La suite des $\inf E_n$ est une suite croissante en ce sens que, pour tout $n$, $\inf E_n \subset \inf E_{n+1}$.

On remarque également que, pour tout $n$, $\inf E_n \subset E_n \subset \sup E_n$.

Enfin, on utilise la notation $\limsup E_n = \bigcap \limits_{n \in \N} \sup E_n = \bigcap \limits_{n \in \N} \bigcup \limits_{k \geq n} E_k$ 

et $\liminf E_n = \bigcup \limits_{n \in \N} \inf E_n = \bigcup \limits_{n \in \N} \bigcap \limits_{k \geq n} E_k$.

Dans le cas où $\limsup E_n = \liminf E_n$, on dit que la suite $E_n$ converge.

Notons que $x \in \limsup E_n \iff \forall n, \ \exists p \geq n/ \, x \in E_p$ et que

$x \in \liminf E_n \iff \exists n/ \, \forall p \geq n, \, x \in E_p$.

En pratique, dire que les $E_n$ convergent signifie que si $x$ appartient à une infinité de $E_p$ alors, à partir d'un certain rang $x$ appartient à tous les $E_p$. C'est la traduction de $\limsup E_n \subset \liminf E_n$.


\section{Tribu, mesure}

\subsection{Tribu}

\subsubsection{Définition}

Une tribu $\mathcal{T}$ sur $E$ est un ensemble de parties de $E$:
\begin{itemize}
\item[$\bullet$] stable par union dénombrable;
\item[$\bullet$] stable par passage au complémentaire;
\item[$\bullet$] qui contient $\emptyset$.
\end{itemize}

À partir de ces trois hypothèses, on en déduit que $\mathcal{T}$ 
\begin{itemize}
\item[$\bullet$] est également stable par intersection dénombrable;
\item[$\bullet$] contient $E$.
\end{itemize}

Un ensemble $E$ muni d'une tribu $\mathcal{T}$ s'appelle un espace mesurable.

\subsubsection{Exemples}

$\mathcal{P}(E)$, $\left\{E;~\emptyset \right \}$ sont des tribus appelées respectivement discrète et grossière.

Si $\mathcal{T}$ est une tribu sur $E$ et si $F \subset E$ alors l'ensemble des $\left \{ T \cap F, \, T \in \mathcal{T}\right \}$ est une tribu sur $F$ appelée tribu trace.


\subsubsection{Tribu engendrée}

Soit $\mathcal{S} \subset \mathcal{P}(E)$. Alors il existe une tribu $\mathcal{T}$ qui contient $\mathcal{S}$ et qui est la plus petite en ce sens que n'importe quelle tribu $\mathcal{T'}$ qui contient $\mathcal{S}$ contient également $\mathcal{T}$.

On construit cette tribu $\mathcal{T}$ par l'extérieur, en posant
\[
\mathcal{T} = \bigcap  \limits_{\substack{\mathcal{T'} \text{ est une tribu}\\ \mathcal{S} \subset \mathcal{T'}}} \mathcal{T'}
\]

Un exemple classique s'appelle la tribu borélienne. C'est la tribu sur $\R$ engendrée par les ouverts.


\subsection{Mesure}

\subsubsection{Définition}

On considère un espace mesurable $\left(E;~\mathcal{T}\right)$. Une mesure $\mu$ est une application de $\mathcal{T}$ dans $\R^{+}$ qui vérifie:
\begin{itemize}
\item[$\bullet$]  $\mu(\emptyset)=0$.
\item[$\bullet$] Pour tout les $(X_i)_{i \in I}$, famille dénombrable d'éléments de $\mathcal{T}$ deux à deux disjoints, \[\mu\left(\bigcup \limits_{i \in I} X_i\right) = \displaystyle{\sum \limits_{i \in I}} \mu(X_i)\]
\end{itemize}

Cette dernière propriété s'appelle la $\sigma-$additivité. Le triplet $\left(E;~\mathcal{T};~\mu\right)$ ainsi construit s'appelle un espace mesuré.

Dans le cas où $\mu(E) < +\infty$, on dit que la mesure est finie.

Dans le cas où $\mu(E) = 1$, on parle de mesure de probabilité.

\subsubsection{Propriétés de la mesure}

Dans tout ce paragraphe, $\left(E;~\mathcal{T};~\mu\right)$ désigne un espace mesuré.

\begin{prop}[Soustraction]
Soient $L$ et $M$ deux éléments de $\mathcal{T}$ tels que $L \subset M$.

Alors
\[
\mu(M-L) + \mu(L)=\mu(M)
\]
\end{prop}


\begin{proof}
Il suffit de constater que $L$ et $M-L$ forment une partition de $M$. Par conséquent, $\mu(L)+\mu(M-L) = \mu(M)$.
\end{proof}


\begin{nota}
On aimerait écrire $\mu(M-L)=\mu(M)-\mu(L)$ mais la soustraction par $\mu(L)$ n'a de sens que si $L$ est de mesure finie. Cette hypothèse est très importante!
\end{nota}

On en déduit la propriété très simple:

\begin{prop}[Inclusion]
Soient $L$ et $M$ deux éléments de $\mathcal{T}$ tels que $L \subset M$. Alors
\[
\mu(M) \geq  \mu(L)
\]
\end{prop}

\begin{prop}[Formule de l'union simple]
Soient $T_1$ et $T_2$ deux éléments de $\mathcal{T}$ tels que $\mu\left(T_1 \cap T_2\right)<+\infty$. 

Alors
\[
\mu\left(T_1 \cup T_2\right)+\mu\left(T_1 \cap T_2\right)=\mu(T_1)+\mu(T_2)
\]
\end{prop}

\begin{proof}
Il suffit de constater que $T_1-\left(T_1 \cap T_2\right)$, $T_2-\left(T_1 \cap T_2\right)$ et $T_1 \cap T_2$ sont trois ensembles disjoints qui vérifient:

\begin{itemize}
\item[$\bullet$] $T_1=\left(T_1-\left(T_1 \cap T_2\right)\right) \cup \left(T_1 \cap T_2\right)$
\item[$\bullet$] $T_2=\left(T_2-\left(T_1 \cap T_2\right)\right) \cup \left(T_1 \cap T_2\right)$
\item[$\bullet$] $T_1 \cup T_2 =\left(T_2-\left(T_1 \cap T_2\right)\right) \cup \left(T_2-\left(T_1 \cap T_2\right)\right) \cup \left(T_1 \cap T_2\right)$
\end{itemize}

En suite, on applique les formules de réarrangement de sommes.
\end{proof}

\begin{prop}[Formule de l'union généralisée]
Soient $\left(T_i\right)_{1 \leq i \leq n}$  des éléments de $\mathcal{T}$ tels que pour tout $1 \leq i < j \leq n$, $\mu\left(T_i \cap T_j\right)<+\infty$. 

Alors:
\begin{multline*}
\mu\left(\bigcup \limits_{1 \leq i \leq n} T_i\right) = \displaystyle{\sum \limits_{1 \leq i \leq n}} \mu(T_i) - \displaystyle{\sum \limits_{1 \leq i_1 < i_2 \leq n}} \mu\left(T_{i_1} \cap T_{i_2}\right) + \cdots + (-1)^{k+1} \displaystyle{\sum \limits_{1 \leq i_1 < i_2 < \cdots < i_k \leq n}} \mu\left( \bigcap \limits_{1 \leq j \leq k} T_{i_j} \right) + \cdots \\
+ (-1)^{n+1} \mu\left( \bigcap \limits_{1 \leq j \leq n} T_j \right)
\end{multline*}
\end{prop}

\begin{proof}
C'est évident pour $n=1$ et $n=2$. On va raisonner par récurrence sur $n$.

Au rang $n+1$, notons que $\bigcup \limits_{1 \leq i \leq n+1} T_i = \left(\bigcup \limits_{1 \leq i \leq n} T_i\right) \cup T_{n+1}$ et utilisons la formule qui précède:
\[
\mu\left(\bigcup \limits_{1 \leq i \leq n+1} T_i\right) = \mu\left(\bigcup \limits_{1 \leq i \leq n} T_i\right)+\mu(T_{n+1})-\mu\left(\bigcup \limits_{1 \leq i \leq n} T_i \cap T_{n+1}\right)
\]

En utilisant l'hypothèse de récurrence, on peut conclure:
\begin{multline*}
\mu\left(\bigcup \limits_{1 \leq i \leq n+1} T_i\right) = \displaystyle{\sum \limits_{1 \leq k \leq n}}(-1)^{k+1}\displaystyle{\sum \limits_{1 \leq i_1 < \cdots < i_k \leq n}}\mu\left(\bigcap \limits_{1 \leq j \leq k} T_{i_j}\right)+\mu(T_{n+1})\\
-\displaystyle{\sum \limits_{1 \leq k \leq n}}(-1)^{k+1}\displaystyle{\sum \limits_{1 \leq i_1 < \cdots < i_k \leq n}}\mu\left(\bigcap \limits_{1 \leq j \leq k} (T_{i_j} \cap T_{n+1})\right)
\end{multline*}

En effet, cette somme correspond à la version éclatée de $\displaystyle{\sum \limits_{1 \leq k \leq n+1}}(-1)^{k+1}\displaystyle{\sum \limits_{1 \leq i_1 < \cdots < i_k \leq n+1}}\mu\left(\bigcap \limits_{1 \leq j \leq k} T_{i_j}\right)$ selon que $i_k=n+1$ ou non.
\end{proof}


Enfin, nous pouvons établir quelques résultats concernant les limites d'ensembles.

\begin{prop}[Convergence monotone, version simple]
Soit $\left(E_n\right)_{n \in N}$ une suite croissante d'éléments de $\mathcal{T}$.

Alors $\lim \uparrow \mu(E_n) = \mu\left( \lim \uparrow E_n\right)$

\medskip
Soit $\left(F_n\right)_{n \in N}$ une suite décroissante d'éléments de $\mathcal{T}$.

On suppose de plus qu'il existe un rang $p$ tel que $\mu(F_p)<+\infty$.

Alors $\lim \downarrow \mu(F_n) = \mu\left( \lim \downarrow F_n\right)$
\end{prop}


\begin{proof}
Pour tout $n$, on pose $I_n = E_{n+1}-E_n$, de sorte que $\lim \uparrow E_n = \bigcup \limits_{n \in N} I_n$. 

Cette famille est ainsi une partition de $\lim \uparrow E_n$.
Ainsi, $\mu\left( \lim \uparrow E_n  \right) = \displaystyle{\sum \limits_{n \in N}} \mu(I_n)$. 

Or, par construction, $\displaystyle{\sum \limits_{1 \leq k \leq n}} \mu(I_k) = \mu(E_n)$ et ainsi $\mu\left( \lim \uparrow E_n  \right) = \lim \uparrow \mu(E_n)$.

\medskip
Intéressons-nous au cas décroissant.

Pour tout $n \geq p$, on considère la suite des $\left(F_p-F_n\right)_{n \geq p}$. C'est une suite croissante et on peut conclure à l'aide du théorème de convergence monotone classique. En effet,

D'une part, $\lim \uparrow \mu\left(F_p-F_n\right)=\mu(F_p) - \lim \downarrow \left(\mu(F_n)\right)$ car $\mu(F_p)<+\infty$.

D'autre part, $\lim \uparrow \mu\left(F_p-F_n\right) = \mu\left(\lim \uparrow (F_p-F_n)\right) = \mu\left(F_p-\lim \downarrow F_n\right) = \mu(F_p)-\mu\left( \lim \downarrow F_n \right)$ en raison de la convergence monotone et de l'hypothèse $\mu(F_p)<+\infty$.

Finalement, on obtient bien $\lim \downarrow \left(\mu(F_n)\right) = \mu\left( \lim \downarrow F_n \right)$
\end{proof}

De la cas décroissant l'hypothèse de mesure finie est importante.

Par exemple, pour la mesure de comptage, si on considère $F_n = \left [0;~ \dfrac{1}{n} \right ]$, pour tout $n$, 

$\mu(F_n) =+\infty$ or $\mu\left(\lim \downarrow F_n\right) = 1$ car $\lim \downarrow F_n = {0}$.

\begin{prop}[Majoration de $\mu\left(\bigcup \limits_{n \in \N} E_n\right)$]
Pour toute famille dénombrable $\left(E_n\right)_{n \in \N}$ de $\mathcal{T}$, telle que, pour tout $n$, $\mu(E_n)<+\infty$, on a:
\[
\mu\left(\bigcup \limits_{n \in \N} E_n\right) \leq \displaystyle{\sum \limits_{n \in \N}} \mu(E_n)
\]
\end{prop}

\begin{proof}
On va le montrer dans le cas discret par récurrence sur le cardinal. 

On considère une famille finie $(E_k)_{1 \leq k \leq p}$. Si $p=1$, le résultat est évident.

Notons maintenant que $\bigcup \limits_{1 \leq k \leq p+1} E_k = \left( \bigcup \limits_{1 \leq k \leq p} E_k \right) \cup E_{p+1}$.

Ainsi,
\[
\mu\left( \bigcup \limits_{1 \leq k \leq p+1} E_k \right) + \mu\left( \left( \bigcup \limits_{1 \leq k \leq p} E_k \right) \cap E_{p+1}  \right) = \mu\left( \bigcup \limits_{1 \leq k \leq p} E_k \right) + \mu(E_{p+1}) 
\]

Finalement, en utilisant l'hypothèse de récurrence:
\[
\mu\left( \bigcup \limits_{1 \leq k \leq p+1} E_k \right) \leq \mu\left( \bigcup \limits_{1 \leq k \leq p} E_k \right) + \mu(E_{p+1}) \leq \displaystyle{\sum \limits_{1 \leq k \leq p}} \mu(E_k) + \mu(E_{p+1})
\]

Par passage à la limite, pour $p \to +\infty$, sachant que la famille $\left( \bigcup \limits_{1 \leq k \leq p} E_k\right)_{p \in \N}$ est croissante, on peut utiliser le théorème de convergence monotone.
\end{proof}

\begin{prop}
On reprend les mêmes hypothèses.

Si $\displaystyle{\sum \limits_{n \in \N}} \mu(E_n)<+\infty$ alors $\mu\left(\limsup E_n\right) = 0$
\end{prop}

\begin{proof}
On suppose que $\displaystyle{\sum \limits_{n \in \N}} \mu(E_n)<+\infty$. 

On pose alors $R_n = \displaystyle{\sum \limits_{p \geq n}} \mu(E_n)$. Par hypothèse, on a donc $R_n \to 0$.

Or, $\mu\left(\bigcup \limits_{p \geq n} E_n\right) \leq R_n$ et ainsi $\lim \downarrow \mu\left(\bigcup \limits_{p \geq n} E_n\right) = 0$.

En utilisant la version décroissante du théorème de convergence monotone on peut conclure.
\end{proof}

\subsubsection{Exemples}

La mesure de comptage, à tout ensemble associe son cardinal.

La mesure de Lebesgue qui à tout intervalle $[a;~b]$ associe $b-a$ constitue également une mesure mais il n'est pas aisé de construire l'extension de la mesure des intervalles à la mesure de n'importe quel élément de la tribu borélienne.

%
%\end{document}