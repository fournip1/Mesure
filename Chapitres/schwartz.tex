
\section{Généralités sur l'espace de Schwartz}

\subsection{Notations}

On se place dans $\R^d$ que l'on munit de la mesure $\tilde{\lambda}_d = \frac{1}{(2\pi)^{n/2}} \lambda_d$ où $\lambda_d$ est la mesure de Lebesgue. Grâce à cette manipulation, le noyau gaussien noté $g: \, x \mapsto \e^{-\norm{x}^2/2}$ vérifie $\norm{g}_1 = 1$.


\medskip
Pour tout $t \in \R^d$, on pose $e_t: \, x \mapsto \e^{\im  x \cdot t}$ où \og $\cdot$ \fg{} désigne le produit scalaire \og classique \fg{} sur $\R^d$.

\medskip
Pour tout $x = (x_1;~x_2;~\cdots;~x_d) \in \R^d$ et pour tout multi-indice $\alpha  = (\alpha_1;~\alpha_2;~\cdots;~\alpha_d) \in \N^d$, on note $x^\alpha = \displaystyle{\prod_{i=1}^d} x_i^{\alpha_i}$.

\medskip
Concernant la dérivation, pour tout multi-indice $\alpha  = (\alpha_1;~\alpha_2;~\cdot;~\alpha_d) \in \N^d$, on note $D^\alpha = \dfrac{\partial^{\alpha_1} \partial^{\alpha_2} \cdots \partial^{\alpha_d}}{\partial x_1^{\alpha_1} \partial x_2^{\alpha_2} \cdots \partial x_d^{\alpha_d}}$.

Et on notera aussi $D_{\alpha} = \frac{1}{(\im)^{\abs{\alpha}}} D^{\alpha}$ où $\abs{\alpha} = \alpha_1 + \alpha_2 + \cdot + \alpha_d$.

\medskip
Ces notations seront utiles lorsque nous établirons un lien entre polynômes, dérivation et transformée de Fourier.

\medskip
Enfin, si $P$ est un polynôme de $\C[X_1,~X_2,\cdots,~X_d]$ dont les coefficients sont $(c_{\alpha})_{\alpha  \in \N^d}$, on notera 

$P(D) = \displaystyle{\sum \limits_{\alpha \in \N^d}} c_{\alpha} D_{\alpha}$.


\subsection{Définition: l'espace de Schwartz}

\begin{de}[Espace de Schwartz]
On dit qu'une fonction $f: \, \R^d \to \C$ appartient à l'espace de Schwartz, noté $\mathcal{S}$, lorsque:
\begin{itemize}
\item[$\bullet$] 
$f \in \mathcal{C}^{\infty}$;
\item[$\bullet$] 
$\forall (\alpha;~p) \in  \N^d \times \N$, $\sup \limits_{x \in \R^d} \left ( 1+\norm{x}^2\right )^{p} \norm{D^{\alpha} f(x)} < +\infty$.
\end{itemize}

Dans ce cas, pour tout $(\alpha;~p) \in  \N^d \times \N$, on notera $N_{\alpha;~p} (f) = \sup \limits_{x \in \R^d} \left ( 1+\norm{x}^2\right )^{p} \norm{D^{\alpha} f(x)}$.
\end{de}

Notons que la famille $N_{\alpha;~p}$ est une famille dénombrable et séparatrice de semi-normes puisque, pour $\alpha = 0$, il s'agit en fait d'une norme. On travaillera donc avec la topologie induite par cette famille, telle qu'elle a été définie dans le document \emph{normes}. 


\begin{cerveau}
On rappelle qu'une base d'ouverts de cette topologie est formée des intersections finies de boules $V_{N_{\alpha;~p};~q_{\alpha;~p}} = \left \{f \in \mathcal{S}/ \, N_{\alpha;~p}(f) < \frac{1}{q_{\alpha;~p}} \right \}$, avec $(\alpha;~p) \in \N^d \times \N$ et  $q_{\alpha;~p} \in \N^*$.
\end{cerveau}


\begin{listremarques}
\item
Notons que $\mathcal{S} \subset \mathcal{C}_0$, l'ensemble des fonctions continues qui tendent vers $0$ en $+\infty$. En effet, toute fonction $f \in \mathcal{S}$ est dominée par $x \mapsto \dfrac{1}{1+\norm{x}^2}$. 

\medskip
On rappelle au passage que $\mathcal{C}_0 \subset \mathcal{C}_b$, les fonctions continues et bornées.

\item
Remarquons aussi que, pour deux entiers $p \geq q$, et pour tout $x \in \R^d$, on a 

$(1+\norm{x}^2)^p \geq (1+\norm{x}^2)^q$, ce qui entraîne, pour tout multi-indice $\alpha$, $N_{\alpha;~p} \geq N_{\alpha;~q}$.

\item
On appelle aussi \emph{fonctions à décroissance rapide} cet ensemble.
\end{listremarques}

\begin{lem}[Majoration grossière pour un polynôme]
Pour tout $x \in \R^d$ et pour tout $\alpha \in \N^d$:
\[
\abs{x^\alpha} = \displaystyle{\prod_{i=1}^d} \abs{x_i}^{\alpha_i} \leq \left (1+\norm{x}^2\right )^{\abs{\alpha}}
\]

\medskip
On en déduit que, si $P$ est un polynôme de degré $q$, il existe un réel positif $M_P$, tel que, pour tout $x \in \R^d$:
\[
\abs{P(x)} \leq M_P (1+\norm{x}^2)^{q}
\]
\end{lem}

\begin{proof}
La première majoration n'est pas difficile à condition de distinguer deux cas. 
\begin{itemize}
\item[$\bullet$] 
Si $\max \limits_{i} \abs{x_i}<1$, dans ce cas $\abs{x^\alpha} < 1$ et donc la majoration est triviale.
\item[$\bullet$] 
Si $\max \limits_{i} \abs{x_i} \geq 1$, dans ce cas, $\norm{x} \geq 1$  et, pour tout $i$, $\abs{x_i} \leq \norm{x} \leq (1+\norm{x}^2)$. On en déduit que $\abs{x^\alpha} \leq (1+\norm{x}^2)^{\abs{\alpha}}$
\end{itemize}

Pour la seconde majoration, remarquons que pour tout multi-indice $\beta$ tel que $\abs{\beta} \leq q$, 

$\abs{x^\beta} \leq (1+\norm{x}^2)^{\abs{\beta}} \leq (1+\norm{x}^2)^{q}$. Ainsi, en posant $K = \max \limits_{\beta} \abs{c_{\beta}}$ avec $c_{\beta}$ les coefficients de $P$, on a la majoration grossière suivante:
\[
\abs{P(x)} \leq \displaystyle{\sum \limits_{\beta \in \N^d}} \abs{c_{\beta}} \abs{x^{\beta}} \leq (1+q)^d K (1+\norm{x})^q
\]

En effet, $(1+q)^d$ est une majoration grossière du nombre de coefficients non nuls.
\end{proof}

Cela nous permet d'en déduire facilement la propriété suivante.

\begin{prop}[Caractérisation équivalente]
Pour toute fonction $f$, on a:

\medskip
$f \in \mathcal{S}$ si et seulement si $f$ est de classe $\mathcal{C}^{\infty}$ et, pour tout $\alpha \in \N^d$ et pour tout polynôme $P$, $P D^{\alpha} f$ est borné.
\end{prop}

Un second lemme qui nous sera utile pour la suite.

\begin{lem}[Formule de Liebnitz en dimension finie]
Soit $\alpha \in \N^d$ et soient $f$ et $h$ deux fonctions définies sur $\R^d$ et à valeurs dans $\C$. 

\medskip
On suppose que $D^{\alpha} f$ et $D^{\alpha} h$ existent.

\medskip
Alors $D^{\alpha} (fh)$ existe. Plus précisément il existe des coefficients entiers naturels $(c_{\beta,~\gamma})_{\beta  + \gamma= \alpha}$ tels que:
\[
D^{\alpha} (fh) = \displaystyle{\sum \limits_{\beta + \gamma = \alpha}} c_{\beta,~\gamma} D^{\beta} f D^{\gamma} h
\]
\end{lem}


\begin{proof}
Ce lemme se montre sans trop de difficulté par récurrence sur $d$, la dimension de l'espace de départ. Je ne le fais pas car les notations multi-indices sont lourdingues. L'idée est simple: appliquer $\dfrac{\partial^{\alpha_1}}{\partial x_1^{\alpha_1}}$ à $\dfrac{\partial^{\alpha_2} \cdots \partial^{\alpha_d}}{\partial x_2^{\alpha_2} \cdots \partial x_d^{\alpha_d}} (fh)$ en appliquant l'hypothèse de récurrence à cette dernière expression.
\end{proof}


\begin{prop}[Propriétés des fonctions de Schwartz]
$\mathcal{S}$ forme une $\C$-algèbre. Muni de la topologie évoquée plus haut, $\mathcal{S}$ est un espace complet. 


\medskip
En outre, pour $h \in \mathcal{S}$, pour $P$ un polynôme et pour $\alpha \in \N^d$, les applications linéaires suivantes sont des endomorphismes continus:
\begin{align*}
f & \mapsto hf &
f & \mapsto Pf &
f & \mapsto D^{\alpha} f
\end{align*}
\end{prop}


\begin{proof}
On sait que $\mathcal{C}^{\infty}$ est une algèbre. Il nous faut donc montrer les propriétés associées aux semi-normes.

\medskip
Il est clair que si $(f;~h) \in \mathcal{S}^2$ et si $\lambda$ et $\mu$ sont des complexes alors, pour tout $(\alpha;~p) \in \N^d \times N$, $N_{\alpha;~p}(\lambda f + \mu h) \leq \abs{\lambda} N_{\alpha;~p}(f) + \abs{\mu} N_{\alpha;~p}(h) < +\infty$ donc $\mathcal{S}$ est bien un espace vectoriel.

\medskip
De plus, sous les mêmes hypothèses, en reprenant les notations de la formule de Liebnitz énoncée plus haut, on a:
\[
N_{\alpha;~p}(fh) \leq  \displaystyle{\sum \limits_{\beta + \gamma = \alpha}} c_{\beta,~\gamma} N_{\beta;~p} (f) N_{\gamma;~p} (h) < +\infty
\]

Cela confirme que $\mathcal{S}$ est une algèbre.

\medskip
Montrons maintenant que $\mathcal{S}$ est complet. Considérons une suite de Cauchy $(f_n)$ d'éléments de $\mathcal{S}$. Pour tout multi-indice $\alpha$, $D^{\alpha} f_n$ est de Cauchy pour la norme infinie (celle de la convergence uniforme).

En particulier, on en déduit que $(f_n)$ converge uniformément vers une fonction $f$ et que $\left (D^{\alpha} f_n\right )$ converge également uniformément vers $D^{\alpha} f$. Et cela prouve au passage que cette limite $f$ est $\mathcal{C}^{\infty}$. Il ne nous reste plus qu'à prouver:
\begin{itemize}
\item[$\bullet$] 
$f \in \mathcal{S}$;
\item[$\bullet$] 
$(f_n)$ converge vers $f$ pour la topologie de $\mathcal{S}$.
\end{itemize}

Soit ainsi $p \in \N$ quelconque et soit $\varepsilon >  0$. Il existe un rang $N$ tel que, pour tout $q \in \N$ et pour tout $n \geq N$, on a $N_{\alpha;~p} (f_{n+q} - f_n) < \varepsilon$. En particulier, pour $q \to +\infty$, on obtient $N_{\alpha;~p} (f - f_n) \leq \varepsilon$, ce qui achève la démonstration. En effet, cela prouve d'une part que  $N_{\alpha;~p} (f) < +\infty$ et d'autre part la convergence pour la topologie de $\mathcal{S}$, quitte à éventuellement réitérer la manipulation pour plusieurs valeurs de $\alpha$ et $p$ et à considérer le maximum des rangs obtenus.

\medskip
D'après les deux lemmes plus haut, il est clair que les trois applications considérées sont des endomorphismes. 

\medskip
La continuité concernant la dérivation est triviale. En effet, pour tout multi-indices $\alpha$ et $\beta$ et, pour tout $p \in \N$, on a $N_{\beta;~p} \left (D^{\alpha} f\right ) = N_{\beta+\alpha;~p} (f)$. La convergence des $(f_n)$ vers $(f)$ dans $\mathcal{S}$ entraîne donc la convergence des $\left (D^{\alpha} f_n\right )$ vers $D^{\alpha} f$.

\medskip
Prouvons maintenant la continuité des deux autres applications considérées. Soit $(f_n)$ une suite de fonctions qui converge vers $f$ dans $\mathcal{S}$. Pour tout $(\alpha;~p) \in \N^d \times \N$, on a, en reprenant les notations de la formule de Liebnitz:
\[
N_{\alpha;~p}(h(f_n-f)) \leq \displaystyle{\sum \limits_{\beta + \gamma = \alpha}} \abs{c_{\beta,~\gamma}} N_{\beta;~p} (f_n-f) N_{\gamma;~0} (h)
\]

On en déduit en particulier que $\lim \limits_{n} N_{\alpha;~p}(h(f_n-f)) = 0$, ce qui prouve que $f \mapsto hf$ est continue.

\medskip
La continuité de $f \mapsto Pf$ est un peu plus pénible. On suppose que $P$ est de degré $q$. Pour tout $x \in \R^d$:
\[
\abs{(1+\norm{x}^2)^p D^{\alpha} [Pf] (x)} \leq (1+\norm{x}^2)^p \displaystyle{\sum \limits_{\beta + \gamma = \alpha}} \abs{c_{\beta,~\gamma}} \abs{D^{\beta} [f] (x)} \abs{D^{\gamma} [P] (x)}
\]

Or $D^{\gamma}[P]$ est un polynôme de degré inférieur ou égal à $q$, ce qui permet, d'après le lemme de majoration grossière, l'existence d'un réel positif $M_{\gamma}$ tel que, pour tout $x$, $\abs{D^{\gamma}[P](x)} \leq M_{\gamma} (1+\norm{x}^2)^q$. On obtient donc:
\[
\abs{(1+\norm{x}^2)^p D^{\alpha} [Pf] (x)} \leq (1+\norm{x}^2)^{p+q} \displaystyle{\sum \limits_{\beta + \gamma = \alpha}} M_{\gamma} \abs{c_{\beta,~\gamma}} \abs{D^{\beta} [f] (x)} 
\]

De cette inégalité, on déduit:
\[
N_{\alpha;~p} (Pf) \leq \displaystyle{\sum \limits_{\beta + \gamma = \alpha}} M_{\gamma} \abs{c_{\beta,~\gamma}} N_{\beta;~p+q} (f)
\]

Ce qui permet de conclure. Pour toute suite $(f_n)$ qui converge vers $f$ dans $\mathcal{S}$, alors $(Pf_n)$ converge vers $Pf$ dans $\mathcal{S}$.
\end{proof}

Énonçons quelques propriétés simples sur l'intégrabilité de fonctions de $\mathcal{S}$.

\begin{prop}[$\mathcal{S}$ et l'intégration]
Soit $f \in \mathcal{S}$. Alors, pour tout $p \in [1;~+\infty]$, et pour tout multi-indice $\alpha$, $D^{\alpha} f \in \L^p$.

\medskip
En outre, $\mathcal{S}$ est dense dans $\L^p$ pour $p \in [1;~+\infty[$.
\end{prop}

\begin{proof}
La première affirmation est facile car $D^{\alpha} f$ est continue (donc mesurable) et dominée par $\frac{1}{1+\norm{x}^2}$ qui est $\L^p$.

\medskip
Remarquons pour la seconde affirmation que les fonctions $\mathcal{C}^{\infty}$ à support compact sont incluses dans $\mathcal{S}$. Or ces premières fonctions sont denses dans $\L^p$\footnote{Nous l'avions montré dans le document \emph{mesure}}.
\end{proof}

\section{Transformée de Fourier dans $\mathbf{\mathcal{S}}$}

\subsection{Définition et premières propriétés}

On s'appuie ici sur les notations établies en introduction.

\begin{de}[Transformée de Fourier]
Soit $f \in \mathcal{S}$. 

\medskip
On pose $\widehat{f}: \, t \mapsto \displaystyle{\int}_{\R^d} \e^{-i t \cdot x} f(x) \, \mathrm d \tilde{\lambda}_d(x) = \displaystyle{\int}_{\R^d} \e_t(x) f(x) \, \mathrm d \tilde{\lambda}_d(x)$. 

\medskip
$\widehat{f}$ est une application de $\R^d$ vers $\C$.
\end{de}

On sait que $\mathcal{S} \subset \L^1$ donc cette application est bien définie. Quelques rappels concernant la transformée de Fourier dans $\L^1$.

\begin{prop}[Transformée de Fourier de fonction $\L^1$]
Pour tout $f \in \L^1$, $\widehat{f} \in \mathcal{C}_0$. En outre, l'application suivante est un morphisme continu:
\[
\begin{array}{llcl}
\psi: \, & \L^1 & \to & \mathcal{C}_0 \\
 & f & \mapsto & \widehat{f}
\end{array}
\]
\end{prop}

\begin{proof}
Nous l'avions prouvé dans le document \emph{mesure.}
\end{proof}

\begin{prop}[Transformée de Fourier du noyau Gaussien]
On rappelle que $g: \, t \mapsto \e^{-\norm{t}^2/2}$ est le noyau Gaussien. Alors:
\[
\widehat{g} = g
\]
\end{prop}

\begin{proof}
Fait dans le document \emph{mesure}, de deux manières différentes.
\end{proof}



\begin{de}[Translation  et dilatation]
Pour tout $x \in \R^d$ et pour tout fonction $f$ définie sur $\R^d$, on pose $\tau_x[f]$ la fonction:
\[
\tau_x[f]: \, t \mapsto f(t-x)
\]

Pour tout $\alpha > 0$, on pose $\delta_{\alpha}[f]$ la fonction:
\[
\delta_{\alpha}[f]: \, t \mapsto f\left (\frac{t}{\alpha}\right )
\]
\end{de}


\begin{prop}[Premières propriétés de la transformée de Fourier]
Soit $P$ un polynôme, $x \in \R^d$ et $f$ et $h$ deux fonctions de $\mathcal{S}$.

Alors:
\begin{align*}
\widehat{\tau_x[f]} & = \e_{-x} \widehat{f} & 
\widehat{\delta_{\alpha}[f]} & = \alpha^n \delta_{\tfrac{1}{\alpha}} \left [\widehat{f}\right ] &
\widehat{P(D)[f]} & = P \widehat{f} \\
\widehat{\e_x f} & = \tau_x\left [ \widehat{f}\right ] &
\widehat{f*h} & = \widehat{f} \widehat{h} &
\widehat{Pf} & = P(-D)\left [\widehat{f} \right ] 
\end{align*}

Enfin, l'application $\psi: \, f \mapsto \widehat{f}$ est un endomorphisme continu sur $\mathcal{S}$.
\end{prop}

\begin{proof}
Dans toute cette démonstration $t \in \R^d$ est quelconque.

\medskip
Petit changement de variable pour la première.
\begin{align*}
\widehat{\tau_x[f]}(t) & = \displaystyle{\int} \e^{-\im t \cdot y} f(y-x) \, \mathrm d \tilde{\lambda}_d(y) = \displaystyle{\int} \e^{-\im t \cdot (z+x)} f(s) \, \mathrm d \tilde{\lambda}_d(z) \\
 & = \e^{-\im t \cdot x} \displaystyle{\int} \e^{-\im t \cdot z} f(s) \, \mathrm d \tilde{\lambda}_d(z) = \e_{-x}(t) \widehat{f}(t)
\end{align*}

Inutile pour montrer la seconde. Cela s'obtient directement:
\begin{align*}
\widehat{\e_x f}(t) & = \displaystyle{\int} \e^{-\im t \cdot y} \e^{\im x \cdot y} f(y) \, \mathrm d \tilde{\lambda}_d(y) = \displaystyle{\int} \e^{-\im (t-x) \cdot y} f(y) \, \mathrm d \tilde{\lambda}_d(y) = \tau_x\left [ \widehat{f}\right ](t)
\end{align*}

Encore un changement de variable pour la troisième:
\begin{align*}
\widehat{\delta_{\alpha}[f]}(t) & = \displaystyle{\int} \e^{-\im t \cdot y} f\left (\tfrac{y}{\alpha} \right ) \, \mathrm d \tilde{\lambda}_d(y) = \alpha^n \displaystyle{\int} \e^{-\im \alpha t \cdot z} f(z) \, \mathrm d \tilde{\lambda}_d(y) = \alpha^n \delta_{\tfrac{1}{\alpha}} \left [\widehat{f}\right ] (t)
\end{align*}

Pour la propriété portant sur le produit de convolution, il faut réaliser un changement de variable puis appliquer le théorème de Fubini-Tonelli:
\begin{align*}
\widehat{f*h}(t) & = \displaystyle{\int}  \e^{-\im t \cdot y}  \displaystyle{\int} f(z) h (y-z) \, \mathrm d \tilde{\lambda}_d(z) \, \mathrm d \tilde{\lambda}_d(y) = \displaystyle{\int}\displaystyle{\int}  \e^{-\im t \cdot y} \e^{-\im t \cdot z} \e^{\im t \cdot z}  f(z) h (y-z) \, \mathrm d \tilde{\lambda}_d(z) \, \mathrm d \tilde{\lambda}_d(y) \\
 & = \displaystyle{\int}\displaystyle{\int} \e^{- \im t \cdot (y-z)} f(z) h(y-z)  \e^{-\im t \cdot z} \, \mathrm d \tilde{\lambda}_d(z) \, \mathrm d \tilde{\lambda}_d(y) = \displaystyle{\int}\displaystyle{\int} \e^{- \im t \cdot u} h(u)  \e^{-\im t \cdot z} f(z) \, \mathrm d \tilde{\lambda}_d(z) \, \mathrm d \tilde{\lambda}_d(u) \\
 & = \left (\displaystyle{\int} \e^{- \im t \cdot u} h(u)  \, \mathrm d \tilde{\lambda}_d(u)\right ) \left (\displaystyle{\int} \e^{-\im t \cdot z} f(z) \, \mathrm d \tilde{\lambda}_d(z) \right ) = \widehat{h}(t)  \widehat{f}(t)
\end{align*}

Maintenant, concernant la propriété portant sur la dérivation, on va réaliser la transformée de Fourier de $\dfrac{\partial}{\partial x_1} f$, la généralisation étant très simple pour $D^{\alpha} f$ et enfin, par combinaison linéaire, pour $P(D)[f]$. 

Ici, il s'agit de réaliser une intégration par parties, en identifiant l'intégrale de Riemann avec l'intégrale de Lebesgue (cas d'une fonction continue et intégrable). En pratique, on écrira, pour tout $y = (y_1,~y_2,~\cdots,~y_d) \in \R^d$, $y = \left (y_1,~\tilde{y}\right )$ avec $\tilde{y} \in \R^{d-1}$, et on utilisera la même convention pour $t$, de sorte que:
\begin{align*}
\widehat{\frac{\partial f}{\partial x_1}}(t) & =  \displaystyle{\int}  \e^{-\im t \cdot y} \frac{\partial f}{\partial x_1} (y) \, \mathrm d \tilde{\lambda}_d(y) \\
 & = \frac{1}{\sqrt{2\pi}} \displaystyle{\int_{\R^{d-1}}} \e^{-\im \tilde{t} \cdot \tilde{y}} \left ( \lim \limits_{N \to +\infty} \displaystyle{\int_{-N}^N} \e^{-\im t_1 y_1} \frac{\partial f}{\partial x_1} \left (y_1,~\tilde{y}\right ) \, \mathrm d y_1 \right ) \, \mathrm d \mu_{d-1} \left ( \tilde{y}\right )
\end{align*}

On traite à part, avec une intégration par parties, l'intégrale portant sur $y_1$:
\[
\displaystyle{\int_{-N}^N} \e^{-\im y_1 t_1} \frac{\partial f}{\partial x_1} \left (y_1,~\tilde{y}\right ) \, \mathrm d y_1 = \left [ \e^{-\im t_1 y_1} f\left (y_1,~\tilde{y}\right ) \right ]_{-N}^N + \im t_1 \displaystyle{\int_{-N}^N} \e^{-\im y_1 t_1} f\left (y_1,~\tilde{y}\right ) \, \mathrm d y_1
\]

On obtient ainsi, compte-tenu de la propriété de décroissance rapide de $f$:
\[
\lim \limits_{N \to +\infty} \displaystyle{\int_{-N}^N} \e^{-\im t_1 y_1} \frac{\partial f}{\partial x_1} \left (y_1,~\tilde{y}\right ) \, \mathrm d y_1 = \im t_1 \displaystyle{\int_{\R}} \e^{-\im y_1 t_1} f\left (y_1,~\tilde{y}\right ) \, \mathrm d y_1
\]

En réinjectant, cela donne:
\[
\widehat{\frac{\partial f}{\partial x_1}}(t) = (\im t_1) \widehat{f}(t)
\]

Compte-tenu de la définition de $D_{\alpha}$, en réitérant ce que l'on fait pour la première variable à des variables quelconques et autant de fois que nécessaire, on obtient bien $\widehat{D_{\alpha} f}(t) = t^{\alpha} \widehat{f}(t)$ et cela se généralise avec $P(D)$, par linéarité!

\medskip
La dernière égalité se montre en appliquant le théorème de convergence dominée de Lebesgue dont on vérifie aisément que les conditions d'application sont réunies.

\medskip
En effet, en reprenant les mêmes notations et supposant que le polynôme considéré est $P = X_1$
\[
\widehat{Pf}(t) = \displaystyle{\int}  y_1 \e^{-\im t \cdot y} f(y) \, \mathrm d \tilde{\lambda}_d(y)  = \displaystyle{\int} \frac{-1}{\im} \frac{\partial}{\partial t_1} \e^{-\im t \cdot y} f(y) \, \mathrm d \tilde{\lambda}_d(y) = \frac{-1}{\im} \frac{\partial}{\partial t_1} \displaystyle{\int} \e^{-\im t \cdot y} f(y) \, \mathrm d \tilde{\lambda}_d(y)
\]

En réitérant l'opération sur autant de variables et autant de fois que nécessaire et en exploitant la linéarité, on retrouve le résultat escompté.

\medskip
Reste maintenant à prouver que la transformée de Fourier est un endomorphisme continu! D'après la dernière égalité prouvée, il est clair que, pour $f \in \mathcal{S}$, $\widehat{f}$ est $\mathcal{C}^{\infty}_b$, c'est à dire l'ensemble des fonctions indéfiniment dérivables et dont toutes les dérivées sont bornées. 


Montrons qu'elle aussi à décroissance rapide. Soit $\alpha \in \N^d$ et $p \in \N$. Posons $Q(x) = (1+\norm{x}^2)^p$ et $P=X^{\alpha}$. 

D'après ce que l'on vient de montrer, $Q(D) P f \in \mathcal{S}$ et $\widehat{Q(D) P f} \in \mathcal{S} = (-1)^{\abs{\alpha}} Q D_{\alpha}\widehat{f}$. Or, nous savons que $\widehat{Q(D) P f \in \mathcal{S}}$ est bornée, ce qui prouve que $\widehat{f}$ est à décroissance rapide.

\medskip
Reste enfin à prouver la continuité. Soit ainsi une suite de fonctions $(f_n)$ qui tend vers $f$ dans $\mathcal{S}$. Pour tout $t$, et pour tout $n$, on a:
\[
\abs{Q D_{\alpha}\widehat{f_n}(t) - Q D_{\alpha}\widehat{f}(t)} = \abs{\widehat{Q(D) P (f_n-f) } (t)}
\]

Posons  $g_n = Q(D) P f_n$ et $g = Q(D) P f$. D'après ce qui précède, on sait que les opération de multiplication par un polynôme et de dérivation sont des endomorphismes continus de $\mathcal{S}$, on en déduit ainsi que $g_n$ tend vers $g$ dans $\mathcal{S}$. Il suffit donc de prouver que, pour $g_n$ tendant vers $g$ avec la topologie de $\mathcal{S}$, $\widehat{g_n-g}$ converge uniformément vers $0$. Mais cela est facile, en effet:
\[
\abs{\widehat{g_n-g}(t)} \leq \displaystyle{\int} \frac{1}{1+\norm{y}^2} (1+\norm{y}^2) \abs{g_n-g}(y) \, \mathrm d \tilde{\lambda}_d(y) \leq K N_{0,~1}(g_n-g)
\]

Avec $K = \displaystyle{\int} \frac{1}{1+\norm{y}^2} \, \mathrm d \tilde{\lambda}_d(y)$. 

\medskip
Finalement, la transformée de Fourier est bien un endomorphisme continu de $\mathcal{S}$.
\end{proof}

\subsection{Théorème d'inversion}

\subsubsection{Approximation de l'unité}

On rappelle la définition du noyau Gaussien, $g: \, t \mapsto \e^{-\norm{t}^2/2}$. On rappelle plus bas quelques résultats minimaux dont on aura besoin pour la suite.

\begin{prop}[Approximation de l'unité]
Pour tout $n$, on note $g_n: \, t \mapsto n^d g(nt)$.

\medskip
Alors, pour toute fonction $f$ de classe $\L^1$, $g_n*f$ tend vers $f$ dans $\L^1$.
\end{prop}


\begin{proof}
Fait dans le document \emph{mesure}.
\end{proof}

\begin{listremarques}
\item
Il y a beaucoup d'autres choses à dire. D'abord, $g_n*f$ est de classe $\mathcal{C}^{\infty}$. D'autre part, dans le cas où $f$ est continue, on aussi convergence uniforme de $g_n*f$ vers $f$.
\end{listremarques}


\subsubsection{Théorème d'inversion}

\begin{de}[Un nouvel opérateur]
Soit $f$ une fonction de classe $\L^1$. Alors, on définit:
\[
\check{f}: \, t \mapsto \widehat{f}(-t)
\]
\end{de}

\begin{listremarques}
\item 
Par composition, cet opérateur est continu et linéaire de $\L^1$ vers $\mathcal{C}_0$, munis de leurs topologies usuelles.
\item 
Cet opérateur est également un endomorphisme continu et linéaire de $\mathcal{S}$.
\end{listremarques}


\begin{theo}[Inversion de la transformée de Fourier]
Soit $f$ une fonction de classe $\L^1$ telle que $\widehat{f}$ est aussi de classe $\L^1$. Alors:
\[
\check{\widehat{f}} = f
\]

En particulier, la transformée de Fourier est un isomorphisme continu de $\mathcal{S}$ dont $g$ est un élément neutre.
\end{theo}

\begin{proof}
Pour tout $x$ et pour tout $n$, calculons, $\check{\widehat{g_n*f}}(x)$, sachant que cette quantité existe puisque $\widehat{g_n*f}$ est de classe $\L^1$. On peut aussi exploiter le théorème de Fubini-Tonnelli:
\begin{align*}
\check{\widehat{g_n*f}}(x) & = \displaystyle{\int} \e^{\im x \cdot y} \displaystyle{\int} \e^{-\im y \cdot t} \displaystyle{\int} g_n(t-z)f(z) \, \mathrm d \tilde{\lambda}_d(z) \, \mathrm d \tilde{\lambda}_d(t) \, \mathrm d \tilde{\lambda}_d(y) \\
 & = \displaystyle{\int} \displaystyle{\int} \displaystyle{\int} \e^{-\im y \cdot (t-x)} g_n(t-z)f(z) \, \mathrm d \tilde{\lambda}_d(t)\, \mathrm d \tilde{\lambda}_d(z) \, \mathrm d \tilde{\lambda}_d(y) \\ 
 & = \displaystyle{\int} \displaystyle{\int} \displaystyle{\int} \e^{-\im y \cdot (z+u-x)} g_n(u)f(z) \, \mathrm d \tilde{\lambda}_d(u) \, \mathrm d \tilde{\lambda}_d(z) \, \mathrm d \tilde{\lambda}_d(y) \\
 & = \displaystyle{\int} \displaystyle{\int} \e^{-\im y \cdot (z-x)} f(z) \displaystyle{\int} \e^{-\im y \cdot u} g_n(u) \, \mathrm d \tilde{\lambda}_d(u) \, \mathrm d \tilde{\lambda}_d(z) \, \mathrm d \tilde{\lambda}_d(y) \text{ on a posé }u=t-z \\ 
 & = \displaystyle{\int} \displaystyle{\int} \e^{-\im y \cdot (z-x)} f(z) \widehat{g_n}(y) \, \mathrm d \tilde{\lambda}_d(z) \, \mathrm d \tilde{\lambda}_d(y) \text{ on sait que }\widehat{g_n}(y) = g\left ( \frac{y}{n}\right ) \\  
 & = \displaystyle{\int} \displaystyle{\int} \e^{-\im (z-x)} f(z) g\left ( \frac{y}{n} \right ) \, \mathrm d \tilde{\lambda}_d(z) \, \mathrm d \tilde{\lambda}_d(y) \\ 
 & = \displaystyle{\int} \displaystyle{\int} \e^{-\im y \cdot v} f(v+x)  g\left ( \frac{y}{n} \right ) \, \mathrm d \tilde{\lambda}_d(v) \, \mathrm d \tilde{\lambda}_d(y) \text{ on a posé }v=z-x \\
 & = \displaystyle{\int}  f(v+x)  \displaystyle{\int} \e^{-\im y \cdot v} g\left ( \frac{y}{n} \right ) \, \mathrm d \tilde{\lambda}_d(y) \, \mathrm d \tilde{\lambda}_d(v) \\ 
 & = \displaystyle{\int}  f(v+x)  \widehat{\delta_n[g]} (v) \, \mathrm d \tilde{\lambda}_d(v) \text{ on rappelle que }\delta \text{ est la dilatation} \\  
 & = \displaystyle{\int}  f(v+x)  n^d g(nv) \, \mathrm d \tilde{\lambda}_d(v) \\
 & = \displaystyle{\int}  f(x-w)  g_n(w) \, \mathrm d \tilde{\lambda}_d(w) \text{ on a posé }w=-v\text{ sachant que }g\text{ est paire}\\
 & = f*g_n(x)
\end{align*}

Ainsi, $\check{\widehat{g_n*f}}$ converge vers $f$ dans $\L^1$. 

\medskip
Il nous faut maintenant prouver que $\widehat{g_n*f}$ tend aussi vers $\widehat{f}$ dans $\L^1$.

On sait que $\widehat{g_n*f} = \widehat{g_n} \widehat{f} = \delta_n[g] \widehat{f}: \, t \mapsto \widehat{t} g\left ( \frac{t}{n}\right )$. Cette fonction est dominée par $\abs{\widehat{f}}$ qui est de classe $\L^1$ et on a convergence simple de $\delta_n[g] \widehat{f}$ vers $\widehat{f}$. 

Par le théorème de convergence dominée, on obtient la convergence de $\widehat{g_n*f}$ vers $\widehat{f}$ dans $\L^1$.

\medskip
Exploitons maintenant la remarque faite sur la continuité de l'opérateur $f \mapsto \check{f}$ pour écrire, dans $\L^1$:
\[
\lim \limits_{n \to +\infty} \check{\widehat{g_n*f}} = \check{\widehat{f}} = f
\]

\medskip
Considérons maintenant $f \in \mathcal{S} \subset \L^1$. Il est clair que $\widehat{f} \in \L^1$ car $\widehat{f} \in \mathcal{S}$. Nous sommes donc dans les conditions d'application du résultat précédent. On sait ainsi que, pour presque tout $x$, $\check{\widehat{f}}(x) = f(x)$. Or ces deux fonctions sont dans $\mathcal{S}$. On en déduit que l'égalité est valable pour tout $x$, ce qui prouve que la transformée de Fourier est un automorphisme continu de $\mathcal{S}$ dont la réciproque est $f \mapsto \check{f}$.
\end{proof}

\section{Étude des suites de mesures finies}

Ce premier paragraphe nous donne des outils pour aborder les suites de mesures finies sous un angle fréquentiel.

\subsection{Transformée de Fourier de mesures finies}


\begin{de}[Transformée de Fourier d'une mesure]
Soit $\mu$ une mesure finie définie sur $\left ( \R^d;~\mathcal{B}\right )$ avec $\mathcal{B}$ la tribu borélienne. On pose:
\[
\begin{array}{llcl}
\widehat{\mu} & \R & \to & \C \\
 & x & \mapsto & \displaystyle{\int} \e^{- \im x \cdot t} \mathrm d \mu(t)
\end{array}
\]
\end{de}

\begin{listremarques}
\item
La fonction $t \mapsto \e^{\im t \cdot x}$ est $\mu$-intégrable car $\mu$ est finie;
\item
On a, pour tout $x$, $\abs{\widehat{\mu}(x)} \leq \widehat{\mu}(0) = \mu\left ( \R^d\right )$.
\end{listremarques}


\begin{theo}[Cette transformée est injective]
\label{injectivite_Fourier}
Soit $\mu$ et $\nu$ deux mesures finies. 

\medskip
On a $\mu = \nu$ si et seulement si $\widehat{\mu} = \widehat{\nu}$.

\medskip
De plus, si $f$ et $g$ sont deux fonctions $\L^1$ et positives, on a aussi $f = g$, $\lambda$-presque partout, si et seulement si $\widehat{f} = \widehat{g}$.
\end{theo}

\begin{proof}
La seconde partie se montre facilement à partir de la première. En effet, il suffit de considérer $\mu$ et $\nu$ les mesures dont $f$ et $g$ sont les fonctions de densité.

\medskip
Pour montrer la première partie de cette proposition, on va exploiter encore une fois les noyaux gaussiens et la densité des fonctions continues à support compact $\mathcal{C}_K$ dans $\L^1$.
\end{proof}


Commençons par établir un lemme.

\begin{lem}[Convolution de mesures, et convergence]
On rappelle que $g$ est le noyau gaussien et que, pour tout $n$, $g_n: \, t \mapsto n^d g(nt)$ constitue une suite approximante de l'unité.

\medskip
Pour toute mesure finie $\mu$, on pose:
\[
g_n * \mu: \, x \mapsto \displaystyle{\int} g_n(x-t) \mathrm d \mu(t) 
\]


\medskip
Alors, pour toute fonction $f$ continue à support compact, on a:
\[
\lim \limits_{n \to +\infty} \displaystyle{\int} f(x) g_n * \mu (x) \, \mathrm d \tilde{\lambda}_d(x) = \displaystyle{\int} f(x) \mathrm d \mu(x)
\]
\end{lem}

\begin{proof}
Il est clair que ces deux intégrales existent et satisfont les conditions d'application du théorème de Fubini-Tonelli. En effet, $g_n * \mu$ est une fonction bornée car $g_n$ est bornée et $\mu$ est finie.

Calculons, pour tout $n$:
\begin{align*}
\displaystyle{\int} f(x) g_n * \mu (x) \, \mathrm d \tilde{\lambda}_d(x) - \displaystyle{\int} f(x) \mathrm d \mu(x) & = \displaystyle{\int} \displaystyle{\int} f(x) g_n(x-t) \, \mathrm d \mu(t) \mathrm d \tilde{\lambda}_d(x) -  \displaystyle{\int} f(x) \displaystyle{\int} g_n(t) \, \mathrm d \tilde{\lambda}_d(t) \mathrm d \mu(x) \\
  & = \underbrace{\displaystyle{\int} \displaystyle{\int} f(u+t) g_n(u) \, \mathrm d \tilde{\lambda}_d(u) \mathrm d \mu(t)}_{\text{on  pose } u = x-t} -  \displaystyle{\int} f(x) \displaystyle{\int} g_n(t) \, \mathrm d \tilde{\lambda}_d(t) \mathrm d \mu(x) \\
  & = \underbrace{\displaystyle{\int} \displaystyle{\int} f(t+x) g_n(t) \, \mathrm d \tilde{\lambda}_d(t) \mathrm d \mu(x)}_{\text{on  pose } x = t  \text{ puis } t = u} -  \displaystyle{\int} f(x) \displaystyle{\int} g_n(t) \, \mathrm d \tilde{\lambda}_d(t) \mathrm d \mu(x) \\  
  & = \displaystyle{\int} \displaystyle{\int} (f(t+x) - f(x)) g_n(t) \, \mathrm d \tilde{\lambda}_d(t) \mathrm d \mu(x)  
\end{align*}

On va maintenant exploiter l'uniforme continuité de $f$. Notons $M = \max(\mu(\R^d);~1)$. Soit $\varepsilon>0$. Il existe $\eta$ tel que pour tout $\abs{t}<\eta$, $\abs{f(t+x) - f(x)} < \frac{\varepsilon}{2M}$. À l'aide de l'inégalité triangulaire, on obtient ainsi:
\begin{multline*}
\abs{\displaystyle{\int} f(x) g_n * \mu (x) \, \mathrm d \tilde{\lambda}_d(x) - \displaystyle{\int} f(x) \mathrm d \mu(x)} \leq \frac{\varepsilon}{2M} \displaystyle{\int} \displaystyle{\int}  \mathbb{1}_{\abs{t}<\eta} g_n(t) \, \mathrm d \tilde{\lambda}_d(t) \mathrm d \mu(x) + 2 \norm{f}_{\infty} \displaystyle{\int} \displaystyle{\int}  \mathbb{1}_{\abs{t} \geq \eta} g_n(t) \, \mathrm d \tilde{\lambda}_d(t) \mathrm d \mu(x) \\
\leq \frac{\varepsilon}{2} + 2 \norm{f}_{\infty} \displaystyle{\int} \displaystyle{\int}  \mathbb{1}_{\abs{t} \geq \eta} g_n(t) \, \mathrm d \tilde{\lambda}_d(t) \mathrm d \mu(x)
\end{multline*}

On sait que $\lim \limits_{n \to +\infty} \displaystyle{\int}  \mathbb{1}_{\abs{t} \geq \eta} g_n(t) \, \mathrm d \tilde{\lambda}_d(t) = 0$ donc il existe un rang $N$ tel que, pour tout $n \geq N$,\\
$\displaystyle{\int} \displaystyle{\int}  \mathbb{1}_{\abs{t} \geq \eta} g_n(t) \, \mathrm d \tilde{\lambda}_d(t) \mathrm d \mu(x) < \frac{\varepsilon}{2\max\left ( 2\norm{f}_{\infty} \mu(\R^d);~1\right )}$. Ce qui donne:
\[
\abs{\displaystyle{\int} f(x) g_n * \mu (x) \, \mathrm d \tilde{\lambda}_d(x) - \displaystyle{\int} f(x) \mathrm d \mu(x)} \leq \varepsilon
\]
\end{proof}

Ce second lemme nous permettra de démontrer le théorème sur l'injectivité.

\begin{lem}
On reprend les notations précédentes. Alors, pour tout $n$:
\[
\check{\widehat{g_n*\mu}} = g_n*\mu
\]

En définissant l'opérateur suivant, pour tout $h$ de $\L^1$ par $\check{h}: \, x \mapsto \displaystyle{\int} \e^{\im x \cdot t} h(t) \, \mathrm d \tilde{\lambda}_d(t)$.
\end{lem}

\begin{listremarques}
\item
On montre assez facilement que $\widehat{g_n * \mu} = \widehat{g_n} \widehat{\mu}$. Mais comme $\widehat{\mu}$ est bornée et $\widehat{g_n} \in \mathcal{S}$, cela nous prouve que $\widehat{g_n * \mu}$ est intégrable.
\end{listremarques}

Montrons maintenant ce second lemme.

\begin{proof}
Pour tout $y$, avec les précautions d'usage pour l'application du théorème de Fubini-Tonelli
\begin{align*}
\check{\widehat{g_n*\mu}}(y) & = \displaystyle{\int} \e^{\im y \cdot x} \displaystyle{\int} \e^{-\im x \cdot t} \displaystyle{\int} g_n(t-u) \, \mathrm d \mu(u) \mathrm d \tilde{\lambda}_d(t) \mathrm d \tilde{\lambda}_d(x) \\
 & = \displaystyle{\int} \e^{\im y \cdot x} \displaystyle{\int} \e^{-\im x \cdot t} \displaystyle{\int} g_n(t-u) \, \mathrm d \mu(u) \mathrm d \tilde{\lambda}_d(t) \mathrm d \tilde{\lambda}_d(x) \\
 & = \displaystyle{\int}  \displaystyle{\int}  \displaystyle{\int} \e^{\im x \cdot (y-t)} g_n(t-u) \,  \mathrm d \tilde{\lambda}_d(t) \mathrm d \tilde{\lambda}_d(x) \mathrm d \mu(u)
\end{align*}

Calculons séparément l'intégrale centrale:
\begin{align*}
\displaystyle{\int} \e^{\im x \cdot (y-t)} g_n(t-u) \,  \mathrm d \tilde{\lambda}_d(t) & = \displaystyle{\int} \e^{\im x \cdot (y-u-v)} g_n(v) \,  \mathrm d \tilde{\lambda}_d(v) \\
 & = \e^{\im x \cdot (y-u)} \displaystyle{\int} \e^{-\im x \cdot v} g_n(v) \,  \mathrm d \tilde{\lambda}_d(v) \\
 & = \e^{\im x \cdot (y-u)} \widehat{g_n}(x) = \e^{\im x \cdot (y-u)} g\left ( \frac{x}{n}\right )
\end{align*}

Ainsi:
\begin{align*}
\check{\widehat{g_n*\mu}}(y) & = \displaystyle{\int}  \displaystyle{\int}  \e^{\im x \cdot (y-u)} g\left ( \frac{x}{n}\right )  \, \mathrm d \tilde{\lambda}_d(x) \mathrm d \mu(u) \\
 & = \displaystyle{\int}  \displaystyle{\int}  \e^{-\im x \cdot (y-u)} g\left ( \frac{x}{n}\right )  \, \mathrm d \tilde{\lambda}_d(x) \mathrm d \mu(u) \text{ car $g$ est paire}\\
  & = \displaystyle{\int}   \widehat{\delta_{n}[g]}(y-u)  \mathrm d \mu(u) \text{ avec }\delta\text{ l'opérateur de dilatation}\\
  & = \displaystyle{\int} n^d g(n(y-u)) \mathrm d \mu(u) = g_n*\mu(y)
\end{align*}
\end{proof}

Voici maintenant la preuve du théorème de l'injectivité de la transformée de Fourier.

\begin{proof}
On suppose ainsi que deux mesures finies $\mu$ et $\nu$ vérifient $\widehat{\mu} = \widehat{\nu}$. Dans ce cas, on a, pour tout $n$:
\[
\widehat{g_n}\widehat{\mu} = \widehat{g_n}\widehat{\nu} \text{ donc }\widehat{g_n * \mu} = \widehat{g_n * \nu}
\]

Ainsi, d'après le second lemme, $g_n*\mu = g_n*\nu$. On en déduit, à l'aide du premier lemme et par passage à la limite que, pour toute fonction $f$ continue à support compact:
\[
\displaystyle{\int} f \mathrm d \mu = \displaystyle{\int} f \mathrm d \nu 
\]

On peut, pour tout compact $K$ fabriquer une suite de fonctions $(f_n)$ continues à support compact, bornée, et qui tendent vers $\mathbb{1}_K$, ce qui prouve, par passage à la limite et application du théorème de convergence dominée, $\mu(K) = \nu(K)$.

\medskip
On conclut enfin avec le théorème des classes monotones pour en déduire $\mu = \nu$.
\end{proof}

\subsection{Différents types de convergence}

Dans toute la suite $d$ désigne un entier naturel. On munit $\R^d$ de la tribu borélienne produit.

\begin{de}[Convergence d'une suite de mesures bornées]
Soit $\left(\mu_n\right)_{n \in \N}$ une suite de mesures bornées sur $\R^d$.

Soit $\mu$ une mesure sur $\R^d$.

On dit que la suite de mesures $(\mu_n)$ converge étroitement (resp. faiblement, vaguement) vers $\mu$ lorsque, pour toute fonction $f$ de $\mathcal{C}_b\left(\R^d,\R\right)$ (resp. $\mathcal{C}_0\left(\R^d,\R\right)$, $\mathcal{C}_K\left(\R^d,\R\right)$):
\[
\lim \limits_{n \to +\infty} \displaystyle{\int} f(x) \mathrm d \mu_n(x) = \displaystyle{\int} f(x) \mathrm d \mu(x)
\]
\end{de}




\begin{listremarques}
\item
Notons que, si $\mu$ n'est pas finie, on ne peut pas avoir de convergence étroite. Pour s'en convaincre, considérer la fonction bornée $\mathbb{1}$.
\item
Dans tout ce qui suit, nous allons établir des liens entre ces différents types de convergence.
\end{listremarques}

\begin{prop}[Lien entre les différents types de convergence]
De la série d'inclusions $\mathcal{C}_b\left(\R^d,\R\right) \supset \mathcal{C}_0\left(\R^d,\R\right) \supset \mathcal{C}_K\left(\R^d,\R\right)$, nous déduisons que la convergence étroite entraîne la convergence faible qui elle-même entraîne la convergence vague.
\end{prop}

\begin{prop}[La limite vague d'une suite de mesures est unique]
Tout est dans le titre!
\end{prop}

\begin{proof}
Supposons qu'il existe deux limites (pour la convergence vague) $\mu$ et $\nu$. Soit $P \subset \R^d$ un pavé compact.

\medskip
Posons, pour tout $p \in \N^*$, $\varphi_p: \, x \mapsto \left ( 1-\mathrm{d}(x;~P)\right )^+$ où $\mathrm d$ désigne la distance.

Le support de $\varphi_p$ est un compact. De plus, $\lim \limits_{p \in \N} \downarrow \varphi_p = \mathbb{1}_P$.

Par passage à la limite sur $n$, on a, pour tout $p$, $\int \varphi_p \mathrm d \mu = \int \varphi_p \mathrm d \nu$.

Par passage à la limite sur $p$, on en déduit $\mu(P) = \nu(P)$. On peut ensuite conclure à l'aide du théorème des classes monotones: $\mu = \nu$.
\end{proof}

Les résultats suivants précisent un peu le lien entre les différents types de convergence.

\begin{theo}[Lien entre convergence faible et étroite]
Réciproquement, si $(\mu_n)$ converge faiblement vers $\mu$ et si $\lim \limits_{n \to +\infty} \mu_n\left(\R^d\right) = \mu(\R^d)$, alors $\mu_n$ converge étroitement vers $\mu$.
\end{theo}

Cette dernière condition s'appelle la convergence des masses totales.

\begin{cerveau}
En pratique, lorsqu'on examine une suite de probabilités, convergence étroite et convergence faible sont équivalentes. Un second résultat nous montrera que la convergence vague sera elle-aussi équivalente.
\end{cerveau}

\begin{proof}
L'inclusion de  $\mathcal{C}_0\left(\R^d,\R\right)$ dans $\mathcal{C}_b\left(\R^d,\R\right)$ rend évident la première implication.

Étudions la réciproque.

Considérons une fonction $f$ continue et bornée. Sans réduire les hypothèses, on peut montrer le résultat en supposant $f$ positive. 

En effet, $f$ est bornée et continue si et seulement si $f^{+}$ et $f^{-}$ le sont.

Et, par définition de l'intégrale, la convergence des intégrales de $f^{+}$ et $f^{-}$ entraînera bien la convergence de l'intégrale de $f$.

Pour tout $q \in \N$, on définit sur $\R$ la fonction $\varphi_q: x \mapsto \begin{cases} 1 \text{ si $x \leq p$}\\ p+1-x \text{ si $x \in ]p;~p+1]$}\\0 \text{ sinon} \end{cases}$.

Cette fonction est continue. On définit également sur $\R^d$ la fonction $\Pi_p: x \mapsto \varphi_q\left(\norm{x}\right)$, peu importe le choix de la norme.

Cette fonction est continue, comme composée de fonctions continues et elle est à support le disque fermé de centre $0$ et de rayon $p+1$ qui est compact. 

En particulier, pour tout $q$ et pour tout $n$, toutes les intégrales de l'égalité suivante sont bien définies et bornées:
\begin{multline*}
\displaystyle{\int} f(x) \mathrm d \mu_n(x) - \displaystyle{\int} f(x) \mathrm d \mu(x) = \displaystyle{\int} \Pi_q(x)f(x) \mathrm d \mu_n(x) - \displaystyle{\int} \Pi_q(x)f(x) \mathrm d \mu(x) + \\
\displaystyle{\int} \left(1-\Pi_q(x)\right)f(x) \mathrm d \mu_n(x) - \displaystyle{\int} \left(1-\Pi_q(x)\right)f(x) \mathrm d \mu(x)
\end{multline*}

En notant $M>0$ un majorant de $f$, on a donc la majoration:
\begin{multline*}
\abs{\displaystyle{\int} f(x) \mathrm d \mu_n(x) - \displaystyle{\int} f(x) \mathrm d \mu(x)} \leq \abs{\displaystyle{\int} \Pi_q(x)f(x) \mathrm d \mu_n(x) - \displaystyle{\int} \Pi_q(x)f(x) \mathrm d \mu(x)} + \\
M \left(\displaystyle{\int} \left(1-\Pi_q(x)\right) \mathrm d \mu_n(x) + \displaystyle{\int} \left(1-\Pi_q(x)\right)\mathrm d \mu(x)\right)
\end{multline*}

On va commencer par majorer $M \left(\displaystyle{\int} \left(1-\Pi_q(x)\right) \mathrm d \mu_n(x) + \displaystyle{\int} \left(1-\Pi_q(x)\right)\mathrm d \mu(x)\right)$.

Notons que $\lim \uparrow \Pi_q = \mathbb{1}$ et comme $\mu$ est bornée, pour tout $\varepsilon>0$, il existe $q$ tel que 

$M  \displaystyle{\int} \left(1-\Pi_q(x)\right)\mathrm d \mu(x) \leq \dfrac{\varepsilon}{4}$. On fixe maintenant cette valeur de $q$.

Et comme $\Pi_q$ est continue à support compact, en raison de la convergence étroite et de la convergence des masses totales, il existe $N$ tel que pour tout $n \geq N$,

$
M \abs{\displaystyle{\int} \left(1-\Pi_q(x)\right) \mathrm d \mu_n(x) - \displaystyle{\int} \left(1-\Pi_q(x)\right)\mathrm d \mu(x)} \leq \dfrac{\varepsilon}{4}
$.

On en déduit, en utilisant la majoration de $M  \displaystyle{\int} \left(1-\Pi_q(x)\right)\mathrm d \mu(x)$ que 

$M \displaystyle{\int} \left(1-\Pi_q(x)\right) \mathrm d \mu_n(x) \leq \dfrac{\varepsilon}{2}$.

Finalement, il nous reste à majorer $\abs{\displaystyle{\int} \Pi_q(x)f(x) \mathrm d \mu_n(x) - \displaystyle{\int} \Pi_q(x)f(x) \mathrm d \mu(x)}$. Mais comme la fonction $\Pi_q \times f$ est continue et à support compact, c'est très simple! Il existe un rang $\tilde{N}$ éventuellement plus grand que $N$, tel que pour tout $n \geq \tilde{N}$,

$\abs{\displaystyle{\int} \Pi_q(x) f(x) \mathrm d \mu_n(x) - \displaystyle{\int} \Pi_q(x) f(x) \mathrm d \mu(x)} \leq \dfrac{\varepsilon}{4}$.

On obtient donc bien:
\[
\abs{\displaystyle{\int} f(x) \mathrm d \mu_n(x) - \displaystyle{\int} f(x) \mathrm d \mu(x)} \leq \dfrac{\varepsilon}{4} + \dfrac{\varepsilon}{2} + \dfrac{\varepsilon}{4} = \varepsilon
\]

Ce qui prouve la convergence étroite de $\mu_n$ vers $\mu$.
\end{proof}

\begin{cerveau}
Pour toute fonction $f$, $\mu$-intégrable, on pose $\mu(f) = \int f \mathrm d \mu$. Notons que c'est une forme linéaire positive et que le théorème de Riesz établit que réciproquement, sous certaines conditions, on peut définir une mesure à partir d'une forme linéaire positive sur un espace de fonctions continue à support compact.
\end{cerveau}

Un premier résultat de convergence étroite.

\begin{prop}[Convergence d'une suite de noyaux gaussiens]
En utilisant la mesure de Lebesgue normalisée, pour toute suite de réels strictement positifs $(\rho_n)$ tendant vers $0$, la suite de mesures $(\mu_n)$ de densités $\frac{1}{\rho_n^d} g_{a,\rho_n}: \, t \mapsto \frac{1}{\rho_n^d} \e^{\tfrac{-\norm{t-a}^2}{2 \rho_n^2}}$ converge étroitement vers $\delta_a$ la mesure de Dirac en $a$.
\end{prop}

\begin{proof}
Notons que $(\mu_n)$ est une suite de probabilités.

\medskip
Fixons $\eta>0$. Par un changement de variable, on a:
\[
\displaystyle{\int} \mathbb{1}_{\norm{x-a} \leq \eta} g_{a,\rho_n}(x) \, \mathrm d \tilde{\lambda}(x) = \displaystyle{\int} \mathbb{1}_{\norm{x} \leq \tfrac{1}{\rho_n} \eta} g(x) \, \mathrm d \tilde{\lambda}(x) \underset{n \to +\infty}{\longrightarrow}  1
\]

Pour toute fonction $f$ bornée et continue, et pour tout $\varepsilon>0$, il existe $\eta>0$ tel que, pour tout $x \in \left [ a-\eta;~a+\eta \right ]$, $f(x) \in \left [ f(a)-\varepsilon;~f(a) + \varepsilon\right ]$, ce qui donne, par découpage de l'intégrale et passage à la limite:
\[
f(a) - \varepsilon \leq \lim \inf  \mu_n\left ( f \right ) \leq \lim \sup \mu_n\left ( f \right ) \leq f(a) + \varepsilon
\]

$\varepsilon$ étant arbitraire on obtient $\lim \mu_n(f) = f(a) = \delta_a(f)$.
\end{proof}

\begin{cerveau}
En fait la suite $\frac{1}{\rho_n^d} g_{a,\rho_n}$ constitue, à une translation près, une suite approximante de l'unité. D'ailleurs, on aurait pu traiter ce lemme en exploitant les convolutions.
\end{cerveau}

On va maintenant établir une petite proposition qui possède de nombreuses conséquences.

\begin{prop}[Convergence faible et famille totale]
Soit $\mathcal{F}$ une famille totale dans $\mathcal{C}_0$. Soit $(\mu_n)$ une suite de mesure.

\medskip
Alors $(\mu_n)$ converge faiblement si et seulement si:
\begin{itemize}
\item[$\bullet$] 
$(\mu_n)$ est bornée;
\item[$\bullet$] 
pour tout $f \in \mathcal{F}$, $\mu_n(f)  = \displaystyle{\int} f  \, \mathrm d \mu_n$ converge.
\end{itemize}
\end{prop}

\begin{listexemples}
\item
Les gaussiennes forment une famille totale de $\mathcal{C}_0$. Voir le document sur le théorème de Stone-Weietrass pour s'en convaincre.
\item
On peut préciser l'exemple précédent. 
On dispose ainsi d'une famille totale dénombrable dans $\mathcal{C}_0$. Il suffit de considérer par exemple les fonctions $g_{a;~b}: \, x \mapsto \e^{-a\norm{x}^2+ b \cdot x}$ avec $a \in \Q^+_*$ et $b \in \Q^d$.
\item
Les fonctions continues à support compact $\mathcal{C}_K$ forment une famille totale de $\mathcal{C}_0$.
\end{listexemples}



\begin{proof}
Nous allons montrer uniquement le sens réciproque, en nous appuyant sur le théorème de Riesz. Posons $M > 0$ un majorant de $\left (\mu_n(\R^d)\right )$.

\medskip
Soit $g \in \mathcal{C}_0$ et $\varepsilon>0$. Il existe une sous-famille finie $(f_i)_{i \in I} \in \mathcal{F}^I$ et des scalaires $(\lambda_i)_{i \in I}$ tous non nuls tels que la combinaison linéaire $\tilde{f} = \displaystyle{\sum \limits_{i \in I}} \lambda_i f_i$ vérifie:
\[
\norm{\tilde{f} - g}_{\infty} <  \frac{\varepsilon}{3M}
\]

Et, pour chaque $i \in I$, il existe $n_i$ tel que, pour tout $p$ et $q$ supérieurs à $n_i$, $\abs{\mu_p(f_i) - \mu_q(f_i)} < \frac{\varepsilon}{3 \abs{\lambda_i} \# I}$ où $\# I$ désigne le cardinal de $I$.

En particulier, par construction, en exploitant la linéarité de l'intégrale et l'inégalité triangulaire, on obtient, pour tout $p$ et $q$ supérieurs à $\max \limits_i n_i$, $\abs{\mu_p \left (\tilde{f}\right ) - \mu_q \left (\tilde{f}\right )} < \frac{\varepsilon}{3}$. Ce qui permet de conclure:
\[
\abs{\mu_p(g) - \mu_q(g) } \leq \abs{\mu_p(g) - \mu_p\left ( \tilde{f} \right )} + \abs{\mu_p \left (\tilde{f}\right ) -  \mu_q\left ( \tilde{f} \right )} + \abs{\mu_q \left (\tilde{f}\right ) - \mu_q \left ( g\right ) } < \varepsilon
\]

La suite $\left (\mu_n(g)\right )$ est de Cauchy donc converge. On définit ainsi, pour tout $g \in \mathcal{C}_0$, la forme linéaire $\mu(g) = \lim \limits_{n} \mu_n(g)$. Par passage à la limite, on vérifie qu'il s'agit d'une forme linéaire positive.

Sachant que $\mathcal{C}_K \subset \mathcal{C}_0$, on peut appliquer le théorème de Riesz et associer à la forme linéaire $\mu$ une unique mesure $\mu$ telle que $\mu(g) = \displaystyle{\int} g \, \mathrm d \mu$.
\end{proof}


\begin{listremarques}
\item
La réciproque devient fausse si on enlève l'hypothèse \og suite bornée. \fg{}

Considérer par exemple la suite de mesure, définie pour tout $n > 0$, par $\mu_n = n \delta_n$ où $\delta_n$ est la mesure de Dirac qui vaut $1$ en $n$ et $0$ partout ailleurs.

En effet, pour toute fonction Gaussienne $g$, $\lim \mu_n(g) = 0$. Or, pour la fonction $f: x \mapsto \begin{cases}
1 \text{ si }x \in [-1;~1] \\
\frac{1}{\abs{x}} \text{ sinon} 
\end{cases}$, on a $\mu_n(f) = 1$ pour tout $n$, ce qui est contradictoire car les Gaussiennes forment une famille totale.
\end{listremarques}

Cette proposition possède un corollaire bien commode.

\begin{cor}[Compacité d'une suite bornée de mesure]
De toute suite $(\mu_n)$ de mesures bornées on peut extraire une sous-suite qui converge étroitement.
\end{cor}



\begin{proof}
On exploite la proposition précédente. Soit $(g_n)_{n \in \N}$ une famille totale dénombrable dans $\mathcal{C}_0$.

\medskip
On construit une suite d'extractrices $(\psi_n)$ par l'algorithme suivant.

\medskip
Pour $n=0$, il existe une extractrice $\varphi_0$ telle que $\left (\mu_{\varphi_0(n)}(g_0)\right )_{n \in \N}$ converge.

On pose $\psi_0 = \varphi_0$.

\medskip
Pour $p \in \N$ quelconque, on suppose que l'on a construit une extractrice $\psi_p$ telle que, pour tout $k \leq p$, la suite $\left (\mu_{\psi_p(n)}(g_k) \right )_{n \in \N}$ converge.

\medskip
Mais on sait qu'il existe une extractrice $\varphi_{p+1}$ telle que la suite $\left (\mu_{\psi_p \circ \varphi_{p+1} (n)}(g_{p+1}) \right )_{n \in \N}$ converge. On pose alors $\psi_{p+1} = \psi_p \circ \varphi_{p+1}$ et $\psi_{p+1}$ vérifie que, pour tout $k \leq p+1$, la suite $\left (\mu_{\psi_{p+1}(n)}(g_k) \right )_{n \in \N}$ converge.

\medskip
On pose ensuite, pour tout $n$, $\tilde{\psi}(n) = \psi_n(n)$. Par construction, pour tout $k \in \N$ et pour tout $n \geq k$, $\tilde{\psi}(n) = \psi_k(p)$ avec $p \geq n$. En particulier, la suite $\left ( \mu_{\tilde{\psi}(n)}(g_k)\right )$ converge. Nous sommes donc dans les conditions d'application de la proposition précédente et pouvons conclure:

La suite $\left ( \mu_{\tilde{\psi}(n)}\right )_{n \in \N}$ converge étroitement.
\end{proof}


\subsection{Convergence de mesures et transformée de Fourier}

Nous allons ici établir un résultat mettant en correspondance la convergence simple de la transformée de Fourier et la convergence étroite d'une suite de mesures.

Voici un premier résultat.

\begin{prop}[Premier résultat de convergence avec Fourier]
Soit $(\mu_n)$ une suite bornée de mesures. Soit $\mu$ une mesure finie.

\medskip
Alors $(\mu_n)$ converge étroitement vers $\mu$ si et seulement si la suite des transformées de Fourier $\left (\widehat{\mu}_n\right )$ converge vers la transformée de Fourier $\widehat{\mu}$ de $\mu$.
\end{prop}

Pour réaliser cette démonstration, on pose:
\begin{itemize}
\item[$\bullet$] 
pour tout $\rho>0$, l'opérateur de dilatation qui, à une fonction $f$, associe $\delta_\rho[f]: \, t \mapsto f\left ( \frac{t}{\rho}\right )$;
\item[$\bullet$] 
pour tout $a \in \R^d$, l'opérateur de translation qui,  à une fonction $f$, associe $\tau_a[f]: \, t \mapsto f\left ( t-a\right )$;
\item[$\bullet$] 
pour tout $a \in \R^d$, la fonction $\e_a: \, t \mapsto \e^{\im a \cdot t}$.
\end{itemize}

On rappelle que pour tout $f$ de classe $\L^1$:
\[
\widehat{\e^a f} = \tau_a\left [\widehat{f}\right ] \text{ et }\widehat{\delta_{\rho} [f]} = \rho^d \delta_{1/\rho}\left [ \widehat{f}\right ]
\]

Enfin, on sait que, pourvu que l'on normalise la mesure de Lebesgue, le noyau Gaussien $g$ vérifie $\widehat{g} = g$.

\begin{proof}
Montrons le sens réciproque de cette proposition. On va ici exploiter la densité des Gaussiennes et reprendre les mêmes hypothèses sur $\rho$ et $a$. 
Posons:
\[
g_{a,\rho}: \, t \mapsto \e^{\tfrac{-\norm{t-a}^2}{2\rho^2}}
\]

Compte-tenu des remarques précédentes, $g_{a,\rho} = \rho^d \widehat{\e_a \delta_{1/\rho}[g]}$, avec $g: \, t \mapsto \e^{-\tfrac{-\norm{t}^2}{2}}$. Pour alléger, posons $\psi_{a,\rho} = \rho^d \e_a \delta_{1/\rho}[g]$. C'est une fonction de l'espace de Schwartz.

\medskip
Pour tout $n$, on a, en exploitant le théorème de Fubini-Tonelli
\begin{align*}
\displaystyle{\int} g_{a,\rho} \, \mathrm d \mu_n & = \displaystyle{\int} \widehat{\psi_{a,\rho}}  \, \mathrm d \mu_n = \displaystyle{\int} \displaystyle{\int} \mathrm d \mu_n(\omega) \, \mathrm d \lambda(t) \, \e^{-\im \omega \cdot t} \psi_{a,\rho} (t) \\
 & = \displaystyle{\int} \mathrm d \lambda(t) \, \psi_{a,\rho} (t) \displaystyle{\int} \mathrm d \mu_n(\omega) \,  \, \e^{-\im \omega \cdot t} \\
 & = \displaystyle{\int} \mathrm d \lambda(t) \, \psi_{a,\rho} (t) \widehat{\mu_n}(t)
\end{align*}

Nous sommes dans les conditions d'application du théorème de convergence monotone. Pour $n$ tendant vers l'infini, cette quantité tend vers:
\[
\displaystyle{\int} \mathrm d \lambda(t) \, \psi_{a,\rho} (t) \widehat{\mu}(t) = \displaystyle{\int} g_{a,\rho} \, \mathrm d \mu
\]

Ce qui prouve la convergence étroite de la suite de mesures $(\mu_n)$.

\medskip
Pour montrer le sens direct, on exploite le fait, pour tout $\omega$, la fonction $t \mapsto \e^{\im t \omega \cdot t}$ est continue et bornée, ce qui donne, en raison de la convergence étroite:
\[
\lim \limits_{n} \displaystyle{\int} \e^{\im t \omega \cdot t} \, \mathrm d \mu_n(t) = \displaystyle{\int} \e^{\im t \omega \cdot t} \, \mathrm d \mu(t) \text{ c'est à dire }\lim \limits_{n} \widehat{\mu_n} (\omega) = \widehat{\mu} (\omega)
\]

\end{proof}


