

Dans toute la suite $E$ désigne un espace de Banach et $[a;~b]$ est un intervalle fermé avec $a<b$.

\section{Définition}

\subsection{Fonctions en escalier}


\begin{de}[Subdivision d'un intervalle fermé bornée, pas]
On dit que $\sigma = (a_i)_{0 \leq i \leq n}$ est un subdivision de $[a;~b]$ lorsque l'on a $a = a_0 < a_1 < a_2 < \cdots < a_n = b$.

Le nombre $\max\left\{a_{i+1}-a_i,~0\leq i \leq n-1\right\}$ s'appelle le pas de la subdivision.
\end{de}


\begin{de}[Fonction en escalier, subdivision adaptée, intégrale de fonction en escalier]
Soit $f: [a;~b] \to E$ une fonction.

On dit que $f$ est en escalier lorsqu'il existe une subdivision $\sigma = (a_i)_{0 \leq i \leq n}$ de $[a;~b]$ telle que, pour tout $0 \leq i \leq n-1$, $f$ est constante sur chacun des $]a_i;~a_{i+1}[$; c'est à dire lorsqu'il existe des éléments de $E$ notés $\left(y_i\right)_{0 \leq i \leq n-1}$ tels que $f_{|]a_i;~a_{i+1}[} = y_i \times \mathbb{1}_{]a_i;~a_{i+1}[}$.

Dans ce cas, on définit l'intégrale de Riemann $f$ sur $[a;~b]$ par
\[
\displaystyle{\int_a^b}  = \displaystyle{\sum \limits_{0 \leq i \leq n-1}} y_i(a_{i+1}-a_i)
\]

Et on dit que $\sigma$ est une subdivision adaptée à $f$ en ce sens que $f$ est constante sur chacun des $]a_i;~a_{i+1}[$. En particulier la valeur de l'intégrale ne dépend pas de la subdivision adaptée choisie.
\end{de}

Pour aborder correctement cette notion, il faut donc détailler les propriétés des subdivisions.

\begin{de}[Finesse de subdivision]
Soient $\sigma$ et $\sigma'$ deux subdivisions de $[a;~b]$. On dit que $\sigma'$ est plus fine que $\sigma$ lorsque $\sigma \subset \sigma'$. La finesse définit une relation d'ordre partielle sur l'ensemble des subdivisions de $[a;~b]$.

Dans ce cas, le pas de $\sigma'$ est inférieur au pas de $\sigma$.
\end{de}


\begin{prop}[Subdivision la plus grossière pour une fonction en escalier]
Soit $f$ une fonction en escalier sur $[a;~b]$. Il existe une subdivision adaptée la plus grossière $\sigma$ avec le sens suivant:

Toute subdivision adaptée à $f$ est plus fine que $\sigma$.

Plus généralement, si $\sigma'$ est une subdivision adaptée à $f$ alors toute subdivision $\sigma''$ plus fine que $\sigma$ est également adaptée.
\end{prop}

\begin{proof}
Soit $\sigma = (a_i)_{0 \leq i \leq n}$ une subdivision adaptée à $f$.

Pour tous les $i \in \intint{1}{n-1}$ tels que $f\left(]a_{i-1};a_i[\right)=f(a_i)=f\left(]a_i;a_{i+1}[\right)$, on supprime $a_i$ de $\sigma$. On obtient ainsi une subdivision $\tilde{\sigma} = (\alpha_i)_{0 \leq i \leq r}$ plus grossière que $\sigma$.

On peut affirmer que toute subdivision $\sigma' = (\beta_i)_{0 \leq i \leq p}$ adaptée à $f$ est plus fine que $\tilde{\sigma}$. Pourquoi? 

On sait que $\alpha_0 \in \sigma'$ et $\alpha_r \in \sigma'$. Supposons maintenant par l'absurde qu'il existe $i \in \intint{1}{r-1}$ tel que $\alpha_i \notin \sigma'$. Dans ce cas, il existe $j \in \intint{0}{p-1}$ tel que $\alpha_i \in ]\beta_j;~\beta_{j+1}[$. Mais dans ce cas, on aurait $f\left(]\alpha_{i-1};~\alpha_i[\right)=f(\alpha_i)=f\left(]\alpha_i;\alpha_{i+1}[\right)$ car $f$ est constante sur $]\beta_j;~\beta_{j+1}[$.

On aboutirait donc, par construction, à une situation absurde.

La dernière partie de la proposition est évidente.
\end{proof}

Cette petite réflexion permet de revenir sur l'unicité de l'intégrale de Riemann d'une fonction en escalier.

\begin{proof}
Soit $\sigma =(a_i)_{0 \leq i \leq n}$ une subdivision adaptée à $f$ avec $(y_i)_{0 \leq i \leq n-1}$ les valeurs de $f$ correspondantes et $\tilde{\sigma} = (\alpha_i)_{0 \leq i \leq r}$ la subdivision adaptée la plus grossière avec $(\widetilde{y}_i)_{0 \leq i \leq r-1}$ les valeurs de $f$ correspondantes.

Alors, 
\[
\displaystyle{\sum \limits_{0 \leq i \leq n-1}} y_i(a_{i+1}-a_i) = \displaystyle{\sum \limits_{0 \leq i \leq r-1}} \widetilde{y}_i(\alpha_{i+1}-\alpha_i)
\]

En effet, pour tout $i \in \intint{0}{r-1}$, on pose $(a_{j_k})_{1 \leq k \leq n_i}$ les éléments de $\sigma$ appartenant à $]\alpha_i;~\alpha_{i+1}]$. On vérifie alors, par construction, que:
\[\displaystyle{\sum \limits_{1 \leq k \leq n_i}} y_{j_k-1}(a_{j_k}-a_{j_{k}-1}) = y_i(\alpha_{i+1}-\alpha_i)\]

En réarrangeant ainsi les termes de la première somme \og par paquets \fg{}, on obtient le résultat escompté.
\end{proof}


\begin{prop}[Propriétés des fonctions en escalier, linéarité de l'intégrale]
Soient $f$ et $g$ deux fonctions en escalier sur $[a;~b]$ et à valeurs dans $E$ un espace de Banach.

Soit $\lambda$ un scalaire.

Alors $f+\lambda g$ est une fonction en escalier.

En particulier l'ensemble des fonctions en escalier est un sous-espace vectoriel des fonctions de $[a;~b]$ dans $E$ et, sur cet espace vectoriel, l'intégrale de Riemann est une application linéaire.

Enfin, une fonction en escalier est bornée.
\end{prop}

\begin{proof}
Si $\sigma_f$ et $\sigma_g$ sont des subdivisions adaptées respectivement à $f$ et à $g$ alors $\sigma_f \cup \sigma_g$ est une subdivision adaptée aux deux fonctions à la fois.

Partant de cette remarque, sur $\sigma_f \cup \sigma_g = (a_i)_{0 \leq i \leq n}$, il est clair que $f+\lambda g$ est en escalier car, pour tout $i \in \intint{0}{n-1}$, $\left(f+\lambda g\right)_{|]a_i;~a_{i+1}[} = f_{|]a_i;~a_{i+1}[} + \lambda g_{|]a_i;~a_{i+1}[}$ et cette restriction est donc constante.

L'application de la formule de l'intégrale à $f + \lambda g$ permet de conclure quant à la linéarité de l'intégrale.

Une fonction en escalier est bien sûr bornée car elle ne prend qu'un nombre fini de valeurs.
\end{proof}

\begin{prop}[Majoration de la norme de l'intégrale d'une fonction en escalier]
Soit $f$ une fonction en escalier à valeurs dans $E$ et $\sigma=(a_i)_{0 \leq i \leq n}$ une subdivision adaptée à $f$ et $(y_i)_{0 \leq i \leq n-1}$ les valeurs de $f$ associées.

Alors
\[
\norm{\displaystyle{\int_a^b} f} \leq \displaystyle{\int_a^b} \norm{f}
\]
et, en particulier:
\[
\norm{\displaystyle{\int_a^b} f} \leq \abs{b-a} \norm{f}_{\infty}
\]
Avec $\norm{f}_{\infty} = \max \limits_{x \in [a;~b]} \norm{f(x)}$.
\end{prop}

\begin{proof}
Ce sont des conséquences assez directes de l'inégalité triangulaire.
\end{proof}

\subsection{Intégrale de fonctions}

\begin{prop}[Fonctions en escalier à valeurs réelles]
Si $f$ et $g$ sont deux fonctions en escalier à valeurs réelles alors 
\begin{itemize}
\item[$\bullet$] $fg$ est en escalier;
\item[$\bullet$] $\min(f,g)$ est en escalier;
\item[$\bullet$] $\max(f,g)$ est en escalier.
\end{itemize}
\end{prop}

\begin{proof}
Cela se prouve très facilement en considérant une subdivision adaptée à $f$ et $g$.
\end{proof}

\begin{de}[Intégrale de fonction]
Soit $f$ une fonction de $[a;~b]$ dans $E$. On dit que $f$ admet une intégrale lorsqu'il existe une suite de fonctions en escaliers $\varphi_n$ à valeurs dans $E$ et une suite de fonctions en escaliers $\psi_n$ à valeurs dans $\R^{+}$ telles que:
\begin{itemize}
\item[$\bullet$] pour tout $n$ et pour tout $x$ de $[a;~b]$, $\norm{\varphi_n(x)-f(x)} \leq \psi_n(x)$;
\item[$\bullet$] $\lim \limits_{n \to +\infty} \displaystyle{\int_a^b} \psi_n = 0$.
\end{itemize}

Dans ce cas, $\lim \limits_{n \to +\infty} \displaystyle{\int_a^b} \varphi_n$ existe et on pose:
\[
\displaystyle{\int_a^b} f = \lim \limits_{n \to +\infty} \displaystyle{\int_a^b} \varphi_n
\]

En particulier cette définition ne dépend pas des suites $\varphi_n$ et $\psi_n$ choisies.
\end{de}

\begin{proof}
On va commencer par prouver la convergence puis l'unicité.

C'est ici qu'intervient la complétude de $E$.

Pour tout $n$ et $p$, 
\[\norm{\displaystyle{\int_a^b} \varphi_{n+p} - \displaystyle{\int_a^b} \varphi_{n}} \leq  \displaystyle{\int_a^b} \norm{\varphi_{n+p}-\varphi_{n}}\]

Or, pour tout $x$, $\norm{\varphi_{n+p}-\varphi_{n}} \leq \norm{\varphi_{n+p}(x)-f(x)}+\norm{f(x)-\varphi_{n}} \leq \psi_{n+p}(x)+\psi_{n}(x)$. Finalement,
\[
\norm{\displaystyle{\int_a^b} \varphi_{n+p} - \displaystyle{\int_a^b} \varphi_{n}} \leq \displaystyle{\int_a^b} \psi_{n} + \displaystyle{\int_a^b} \psi_{n+p}
\]

Cette inégalité permet de prouver que la suite $\displaystyle{\int_a^b} \varphi_{n}$ est de Cauchy puisque la suite $\displaystyle{\int_a^b} \psi_{n}$ tend vers $0$.

Reste à prouver maintenant l'unicité.

On considère donc $\tilde{\varphi}_n$ et $\tilde{\psi}_n$ un couple de suites de fonctions vérifiant les deux hypothèses de la définition.

Par des techniques semblables, on peut prouver, pour tout $n$, que
\[
\norm{\displaystyle{\int_a^b} \varphi_{n} - \displaystyle{\int_a^b} \widetilde{\varphi}_{n}} \leq \displaystyle{\int_a^b} \psi_{n} + \displaystyle{\int_a^b} \widetilde{\psi}_{n}
\]

Et là encore, on peut conclure quant à l'unicité de la limite!


\end{proof}

\begin{prop}[Propriétés des fonctions qui admettent une intégrale de Riemann]
Ces fonctions forment un sous-espace vectoriel des fonctions bornées de $[a;~b]$ dans $E$. 

De plus, l'intégrale est une application linéaire sur cet espace.
\end{prop}

\begin{proof}
Cela se prouve très facilement avec quelques inégalités triangulaires.

Ainsi, si $f_1$ et $f_2$ admettent des intégrales, que l'on associe à $f_1$ les suites de fonctions en $\varphi_n^{(1)}$ escalier et $\psi_n^{(1)}$, que l'on associe à $f_2$ les suites de fonctions en $\varphi_n^{(2)}$ escalier et $\psi_n^{(2)}$, on peut vérifier que les suites $\varphi_n^{(1)} + \lambda \varphi_n^{(2)}$ et $\psi_n^{(1)} + \norm{\lambda} \psi_n^{(2)}$ vérifient pour tout $x$ de $[a;~b]$
\[\norm{(f_1 + \lambda f_2)(x) - \left(\varphi_n^{(1)} + \lambda \varphi_n^{(2)}\right)(x)} \leq \left(\psi_n^{(1)} + \norm{\lambda} \psi_n^{(2)}\right)(x)\]
et qu'on a bien $\lim \limits_{n \to +\infty} \displaystyle{\int_a^b} \left(\psi_n^{(1)} + \norm{\lambda} \psi_n^{(2)}\right) = 0$.

Par passage à la limite et grâce à la linéarité de l'intégrale sur les fonctions en escalier, on peut conclure.

Pour finir, il est clair que, pour tout $x$,

$\norm{f(x)} \leq \norm{\varphi_0(x)} + \psi_O(x)$, ce qui prouve qu'une fonction qui admet une intégrale est minorée en norme par une fonction bornée de $\R^{+}$. Elle est donc bornée.
\end{proof}

Un lemme qui sera utile pour la suite.

\begin{lem}[Construction d'une suite \og optimale \fg{} de fonctions en escaliers]
Soit $f$ une fonction intégrable au sens de Riemann sur un intervalle $[a;~b]$.

Alors, il est possible de construire les suites $\varphi_n$ et $\psi_n$ de fonctions en escalier telle sorte que 
\begin{itemize}
\item[$\bullet$] pour tout $n$, et pour tout $t \in [a;~b], \, \norm{f(t)-\varphi_n(t)} \leq \psi_n(t)$;
\item[$\bullet$] $\psi_n$ est la plus petite fonction en escalier compte-tenu de $\varphi_n$;
\item[$\bullet$] $\psi_n$ décroit vers une fonction d'intégrale nulle.
\end{itemize}

En particulier, les suites $\psi_n$ et $\varphi_n$ sont bornées.
\end{lem}


\begin{proof}
On considère une suite $\varphi_n$ et $\psi_n$ vérifiant les hypothèses de la définition de l'intégrale.

On sait que $f$ est bornée. 

Au rang $0$: on considère $\sigma_0$ la subdivision la plus grossière adaptée à $\varphi_0$. 

Sur chacun des sous-intervalles ouverts de $\sigma_0$, on détermine la valeur de $\tilde{\psi}_0$ comme étant la borne supérieure de $\norm{f-\varphi_0}$. Enfin, on pose $\tilde{\varphi_0} = \varphi_0$. Sur les points de la subdivision, on définit $\tilde{\phi}_0 = f$.

On suppose construite ainsi la suite jusqu'à un certain rang $n$.

Au rang $n+1$, on considère la subdivision $\sigma_{n+1}$ la pus grossière adaptée à $\varphi_{n+1}$ et $\tilde{\phi}_n$.

Sur chacun des sous-intervalles de cette subdivision, on choisit la valeur de $\tilde{\varphi}_{n+1}$ entre $\varphi_{n+1}$ et $\tilde{\phi}_n$ de telle sorte que la borne supérieure de $\norm{f-\tilde{\phi}_{n+1}}$ soit minimale et on pose $\tilde{\psi}_{n+1}$ cette borne supérieure.

Par construction, on dispose donc d'une suite $\tilde{\psi}_n$ inférieure à $\psi_n$ et d'une suite $\tilde{\varphi}_n$ qui vérifient les hypothèses de la définition de l'intégrale et qui satisfont les conditions du lemme!
\end{proof}

\begin{theo}[Mesure de Lebesgue et intégrale de Riemann]
Soit $f$ une fonction intégrable au sens de Riemann et à valeurs dans $E$.

Alors, $f$ est presque partout limite simple d'une suite de fonctions en escaliers. 

En particulier, si $E= \R$ ou  $\C$, $f$ est alors mesurable au sens de Lebesgue. 
\end{theo}

\begin{proof}
On reprend les notations et hypothèses de l'intégrabilité au sens de Riemann.

Notons qu'une fonction en escalier à valeurs réelles est mesurable, comme combinaisons linéaires de fonctions mesurables. D'autre part,il est clair que l'intégrale de Riemann d'une telle fonction coïncide avec l'intégrale de Lebesgue.

En particulier, on peut utiliser le théorème de convergence monotone et on obtient:
\[
\lim \downarrow \displaystyle{\int_a^b} \psi_n = \lim \downarrow \displaystyle{\int_{[a;~b]}} \psi_n(x) \, \mathrm d \lambda(x) = \displaystyle{\int_{[a;~b]}} \lim \downarrow \psi_n(x) \, \mathrm d \lambda(x)
\]

Par conséquent, la fonction $\lim \downarrow \psi_n$ est presque partout nulle.

Mais comme, pour tout $x$, $\norm{\varphi_n(x)-f(x} \leq \psi_n(x)$, on en déduit la convergence simple presque partout.

Dans le cas réel, cette remarque nous montre que $f$ est mesurable.
\end{proof}

\begin{prop}[Produit de fonctions intégrables]
Soit $f$ et $\theta$ deux fonctions définies sur $[a;~b]$ à valeurs respectivement dans $E$ et $\R$.

Si $f$ et $\theta$ sont intégrables au sens de Riemann alors $\theta f$ l'est aussi.
\end{prop}

\begin{proof}
Soient $\left(\varphi_n,~\psi_n\right)$ et $\left(\rho_n,~\chi_n\right)$ les suites de fonctions en escaliers associées aux intégrales de $f$ et $\theta$. Pour tout $n$ et pour tout $t \in [a;~b]$:
\[
\norm{\theta(t)f(t)-\rho_n(t)\varphi_n(t)} \leq \norm{\theta(t)f(t)-\rho_n(t)f(t)}+ \norm{\rho_n(t)f(t)-\rho_n(t)\varphi_n(t)} \leq \norm{f(t)} \times \chi_n(t) + \norm{\rho_n(t)} \times \psi_n(t)
\]

On peut conclure car on sait que $f$ et $\rho_n$ sont bornées, indépendamment de $n$ et $t$.
\end{proof}

\section{Applications}

\subsection{Comparaison séries-intégrales}

\begin{lem}[Règle de Riemann sur les séries à termes positifs]
Soit $\displaystyle{\sum \limits_{n \in \N}} a_n$ une série à termes positifs.

S'il existe $\alpha>1$ tel que $n^{\alpha}a_n \underset{n \to +\infty}{\longrightarrow} 0$ alors la série converge.

S'il existe $0 \leq \alpha < 1$ tel que $n^{\alpha}a_n \underset{n \to +\infty}{\cancel{\longrightarrow}} 0$ alors la série diverge.
\end{lem}

On rappelle que l'on connaît la nature des séries de Riemann $\sum \frac{1}{n^\alpha}$ grâce à l'étude de la série $\sum \frac{1}{n}$ et grâce aux comparaisons séries-intégrales.

\begin{proof}
Le cas $\alpha>1$ est facile car on a alors $a_n = o\left ( \frac{1}{n^{\alpha}}\right )$.

On va s'intéresser au cas $0 \leq \alpha < 1$ et supposer $n^{\alpha}a_n \underset{n \to +\infty}{\cancel{\longrightarrow}} 0$. En particulier, il existe $\eta>0$ et une extractrice $\varphi$ telle que $\varphi(n)^{\alpha} \times a_{\varphi(n)} \geq \eta$, ce qui donne $a_{\varphi(n)} \geq \frac{\eta}{\varphi{n}^{\alpha}}$. Or la série des $\sum \frac{1}{n^{\alpha}}$ tend vers $+\infty$, ce qui permet de conclure!
\end{proof}


\subsection{Intégrale de Riemann et suites}

\begin{theo}[Convergence]
Soit une suite de fonctions $f_n$ définies sur un intervalle $[a;~b]$ et à valeurs dans un espace de Banach $E$.

On suppose que:
\begin{itemize}
\item[$\bullet$] les $f_n$ sont Riemann intégrables;
\item[$\bullet$] la suite $f_n$ converge uniformément vers une fonction $f$;
\item[$\bullet$] $f$ est Riemann intégrable.
\end{itemize}

Alors
\[
\lim \limits_{n \to +\infty} \displaystyle{\int_a^b} f_n = \displaystyle{\int_a^b} f
\]
\end{theo}


\begin{proof}
Soit $\varepsilon > 0$. Il existe $N$ tel que, pour tout $n \geq N$, $\norm{f_n-f}_{\infty} < \dfrac{\varepsilon}{b-a}$, ce qui entraîne:
\[
\norm{\displaystyle{\int_a^b} f_n-\displaystyle{\int_a^b} f} \leq (b-a) \norm{f_n-f}_{\infty} < \varepsilon
\]
\end{proof}

Une conséquence de ce théorème concerne la dérivation sous le signe intégrale.

\begin{theo}[Dérivation et intégrale de Riemann]
Soit $I$ un intervalle ouvert et $f$ une fonction définie sur $[a;~b] \times I$ à valeurs dans $\R$.

On suppose que:
\begin{itemize}
\item[$\bullet$] pour tout $y$ de $I$, la fonction $x \mapsto f(x;~y)$ est Riemann intégrable;
\item[$\bullet$] pour tout $x$ de $[a;~b]$, la fonction $y \mapsto f(x;~y)$ est dérivable;
\item[$\bullet$] pour tout $y$ de $I$, la fonction $y \mapsto \dfrac{\partial f(x;~y)}{\partial y}$ est Riemann intégrable;
\item[$\bullet$] la fonction $(x;~y) \mapsto \dfrac{\partial f(x;~y)}{\partial y}$ est uniformément continue sur $[a;~b] \times I$.
\end{itemize}

Alors, la fonction $y \mapsto \displaystyle{\int_a^b} f(x;~y) \mathrm d x$ est dérivable et sa dérivée est $y \mapsto \displaystyle{\int_a^b} \dfrac{\partial f(x;~y)}{\partial y} \mathrm d x$.
\end{theo}

\begin{proof}
On pose $\varphi: y \mapsto \displaystyle{\int_a^b} f(x;~y) \mathrm d x$.

Pour tout $y$ de $I$, on considère une suite $h_n$ de réels non nuls tels que, pour tout $n$, $h_n+y \in I$ et $\lim \limits_{n \to +\infty} h_n = 0$.

Alors, pour tout $n$, $\dfrac{\varphi(y+h_n)-\varphi(y)}{h_n} -  \displaystyle{\int_a^b} \dfrac{ \partial f(x;~y)}{\partial y} \mathrm d x = \displaystyle{\int_a^b} \left(\dfrac{f(x;~y+h_n)-f(x;~y)}{h_n} - \dfrac{\partial f(x;~y)}{\partial y} \right)\mathrm d x$ en raison de la linéarité de l'intégrale.

On conclut ensuite grâce à l'uniforme continuité de la dérivée partielle.
\end{proof}

\subsection{Le cas des dimensions finies}

On rappelle qu'en dimension finie, toutes les normes sont équivalentes.

\subsubsection{Le cas réel}

On étudie ici le cas où l'espace image est $\R$. On dispose alors d'une formulation équivalente beaucoup plus commode.

\begin{lem}[Premier critère d'intégration au sens de Riemann pour les fonctions à valeur réelle]
Une fonction $f: [a;~b] \to \R$ est intégrable au sens de Riemann si et seulement s'il existe deux suites de fonctions en escalier $\left(\varphi_n^b\right)_{n \in \N}$ et $\left(\varphi_n^h\right)_{n \in \N}$ telles que:
\begin{itemize}
\item[$\bullet$]  pour tout $x$ de $[a;~b]$, et pour tout $n$, $\varphi_n^b(x) \leq f(x) \leq \varphi_n^h(x)$;
\item[$\bullet$]  la suite $\varphi_n^b$ est croissante et la suite $\varphi_n^h$ est décroissante;
\item[$\bullet$]  $\lim \limits_{n \to +\infty} \displaystyle{\int_a^b} \left(\varphi_n^h - \varphi_n^b\right) = 0$
\end{itemize}

\end{lem}

\begin{proof}
On reprend les conditions d'existence d'intégrale avec les mêmes notations.

On a alors pour tout $n$ et tout $x$, $\varphi_n(x) - \psi_n(x) \leq f(x) \leq \varphi_n(x) + \psi_n(x)$.

On pose ainsi $\varphi_0^h(x) = \varphi_0(x) + \psi_0(x)$ et $\varphi_0^b(x) = \varphi_0(x) - \psi_0(x)$.

On suppose que l'on a construit jusqu'à un certain rang $n$ la suite des $\varphi_n^h$ et $\varphi_n^b$. Au rang $n+1$, il suffit de poser 

$\varphi_{n+1}^h = \min\left(\varphi_n^h,~\varphi_{n+1}+ \psi_{n+1}\right)$ et $\varphi_{n+1}^b = \max\left(\varphi_n^b,~\varphi_{n+1} - \psi_{n+1}\right)$ qui sont des fonctions en escalier.

On fabrique ainsi par récurrence les suites $\varphi_n^h$ et $\varphi_n^b$ et ces suites valident par construction le sens direct du lemme.

Le sens réciproque est évident. Il suffit pour cela de poser $\psi_n = \varphi_n^h - \varphi_n^b$.
\end{proof}

De plus, dans le cas réel, l'intégrale de Riemann est positive.

\begin{prop}[Positivité de l'intégrale]
Soient $f$ et $g$ deux fonctions Riemann intégrables sur $[a;~b]$ et à valeurs réelles.

Si, pour tout $x \in [a;~b]$, $f(x) \leq g(x)$ alors $\displaystyle{\int_a^b} f \leq \displaystyle{\int_a^b} g$.
\end{prop}

\begin{proof}
Cela se prouve en intégrant $g-f$ qui est positive et en utilisant la linéarité.
\end{proof}

\begin{de}[Sommes de Darboux de $f$ associées à une subdivision]
Soit $f$ une fonction de $[a;~b]$ dans $\R$ bornée.

Soit $\sigma = (a_i)_{0 \leq i \leq n}$ une subdivision de $[a;~b]$.

Pour tout $i \in \intint{0}{n-1}$, on pose 
\begin{align*}
M_i & = \sup \limits_{]a_i;~a_{i+1}[} f\\
\overline{M_i} & = \sup \limits_{[a_i;~a_{i+1}[} f\\
m_i & = \inf \limits_{]a_i;~a_{i+1}[} f\\
\overline{m_i} & = \inf \limits_{[a_i;~a_{i+1}[} f\\
\end{align*}

On définit alors 
\begin{align*}
S_\sigma^h & = \displaystyle{\sum \limits_{i \in \intint{0}{n-1}}} M_i \times \mathbb{1}_{]a_i;~a_{i+1}[} +  \displaystyle{\sum \limits_{i \in \intint{0}{n}}} f(a_i) \times \mathbb{1}_{a_i} \\
S_\sigma^d & = \displaystyle{\sum \limits_{i \in \intint{0}{n-1}}} m_i \times \mathbb{1}_{]a_i;~a_{i+1}[} +  \displaystyle{\sum \limits_{i \in \intint{0}{n}}} f(a_i) \times \mathbb{1}_{a_i} \\
\overline{S_\sigma^h} & = \displaystyle{\sum \limits_{i \in \intint{0}{n-1}}} \overline{M_i} \times \mathbb{1}_{[a_i;~a_{i+1}[} +  \displaystyle{\sum \limits_{i \in \intint{1}{n}}} f(a_i) \times \mathbb{1}_{a_i} \\
\overline{S_\sigma^d} & = \displaystyle{\sum \limits_{i \in \intint{0}{n-1}}} \overline{m_i} \times \mathbb{1}_{[a_i;~a_{i+1}[} +  \displaystyle{\sum \limits_{i \in \intint{1}{n}}} f(a_i) \times \mathbb{1}_{a_i} 
\end{align*}

\end{de}

\begin{proof}
L'existence de ces sommes est garantie par le fait que $f$ soit bornée.
\end{proof}

Cette construction justifie la proposition suivante

\begin{prop}[Propriété fondamentale des sommes de Darboux]
On reprend les mêmes notations et hypothèses que dans la définition.

Pour toutes fonctions en escaliers $\varphi^h$ et $\varphi^b$ associées à $\sigma$ et telles que $\varphi^b \leq f \leq \varphi^h$, on a
\[
\varphi^b \leq S_\sigma^b  \leq f \leq S_\sigma^h \leq \varphi^h
\]
\end{prop}


\begin{proof}
Pour tout $i \in \intint{0}{n-1}$, et pour tout $x \in ]a_i;~a_{i+1}[$, on a

$y_i^b \leq f(x)$ et $f(x) \leq y_i^h$ où $y_i^b$ et $y_i^h$ représentent les valeurs (constantes) de $\varphi^b$ et $\varphi^h$ sur l'intervalle considéré.

Par passage à la borne inférieure sur $f(x)$ dans la première inégalité et à la borne supérieure dans la seconde, on obtient le résultat escompté.
\end{proof}

Par construction les sommes de Darboux réalisent un encadrement optimal de $f$ par des fonctions en escalier pour une subdivision donnée. On en déduit ainsi le corollaire.

\begin{cor}[Sommes de Darboux et encadrement d'intégrale]
On reprend les mêmes notations. Pour toute subdivision $\sigma$ et pour toute fonction $f$ intégrable:
\[
\displaystyle{\int_a^b}  S_\sigma^b  \leq \displaystyle{\int_a^b} f \leq \displaystyle{\int_a^b} S_\sigma^h
\]

En particulier:
\[
\left(\inf \limits_{[a;~b]} f\right) \times (b-a) \leq \displaystyle{\int_a^b} f  \leq \left(\sup \limits_{[a;~b]} f\right) \times (b-a)
\]
\end{cor}

\begin{prop}[Sommes de Darboux et subdivisions plus fines]
On reprend les mêmes hypothèses.

Soit $\sigma \subset \sigma'$ deux subdivisions.

Alors:
\[
S_\sigma^b \leq S_{\sigma'}^b \leq f \leq S_{\sigma'}^h \leq S_{\sigma}^h
\]
et
\[
\overline{S_\sigma^b} \leq \overline{S_{\sigma'}^b} \leq f \leq \overline{S_{\sigma'}^h} \leq \overline{S_{\sigma}^h}
\]
\end{prop}

\begin{proof}
C'est évident par construction des sommes de Darboux.
\end{proof}

\begin{theo}[Convergence des sommes de Darboux vers l'intégrale]
Soit $f$ une fonction bornée de $[a;~b]$ dans $\R$.

$f$ est intégrable au sens de Riemann si et seulement si, pour tout $\varepsilon > 0$, il existe un nombre $\eta > 0$ tel que, pour toute subdivision $\sigma$ de pas inférieur à $\eta$, 
\[
\displaystyle{\int_a^b} \left(S_\sigma^h - S_\sigma^b\right) < \varepsilon
\]
\end{theo}

\begin{proof}
Le sens réciproque est évident.

Montrons le sens direct. Ainsi, on suppose $f$ intégrable au sens de Riemann.

On va considère deux suites de fonctions en escaliers $\varphi_n^h$ et $\varphi_n^b$ telles qu'exposées dans le lemme du début de ce paragraphe.

Soit $\varepsilon > 0$ un nombre.

On sait qu'il existe $n$ tel que $\displaystyle{\int_a^b} \left(\varphi_n^h-\varphi_n^b\right) < \dfrac{\varepsilon}{2}$

Soit $\sigma=(a_i)_{0 \leq i \leq p}$ la subdivision adaptée à ces deux fonctions en escalier.

On pose enfin $M$ un majorant de $\abs{\varphi_n^b}$ et $\abs{\varphi_n^h}$ sur $[a;~b]$.

Pour tout intervalle, $]c;~d[ \subset ]a;~b[$, on sait que  

$\displaystyle{\int_c^d} \left(\sup \limits_{]c;~d[} f -\inf \limits_{]c;~d[} f\right) \leq 2M(d-c)$.

Fort de cette remarque, et du fait que $\varphi_n^h$ et $\varphi_n^b$ contiennent au plus $p+1$ sauts, on va poser $\eta = \dfrac{\varepsilon}{4M(p+1)}$.

Pour toute subdivision $\sigma'=(\alpha_i)_{0 \leq i \leq q}$ de pas plus fin que $\eta$, la majoration de 
$
\displaystyle{\int_a^b} \left(S_{\sigma'}^h - S_{\sigma'}^b\right)
$ se fait en distinguant deux cas sur les $i \in \intint{0}{q-1}$:
\begin{itemize}
\item[$\bullet$] s'il existe $j \in \intint{0}{p-1}$ tel que $]\alpha_i;~\alpha_{i+1}[ \subset ]a_j;~a_{j+1}[$, alors

$\displaystyle{\int_{\alpha_i}^{\alpha_{i+1}}} \left(S_{\sigma'}^h - S_{\sigma'}^b\right) \leq \displaystyle{\int_{\alpha_i}^{\alpha_{i+1}}} \left(\varphi_n^h-\varphi_n^b\right)$

\item[$\bullet$] s'il existe $j \in \intint{1}{p-1}$ tel que $a_j \in ]\alpha_i;~\alpha_{i+1}[$, alors

$\displaystyle{\int_{\alpha_i}^{\alpha_{i+1}}} \left(S_{\sigma'}^h - S_{\sigma'}^b\right) \leq 2 M \left(\alpha_{i+1}-\alpha_{i+1}\right) \leq 2M \eta  = \dfrac{\varepsilon}{2(p+1)}$ 
\end{itemize}

Mais on rappelle qu'il y a au plus $p+1$ discontinuités, correspondant au cas \no2.

En sommant sur tous les $i \in \intint{0}{q-1}$, on obtient la majoration recherchée:
\[
\displaystyle{\int_a^b} \left(S_{\sigma'}^h - S_{\sigma'}^b\right) \leq \dfrac{\varepsilon}{2} + \dfrac{\varepsilon}{2}
\]
\end{proof}

On peut prouver que ce théorème s'étend aussi aux sommes de Darboux \og étendues \fg{} $\overline{S_\sigma^h}$ et $\overline{S_\sigma^b}$.

Une des conséquences de ce théorème concerne les sommes de Riemann, sujet sur lequel nous ne nous étendrons pas.

\begin{de}[Sommes de Riemann, subdivisions pointées]
Une subdivision pointée d'un intervalle $[a;~b]$ est la donnée d'une subdivision $\sigma = (a_i)_{0 \leq i \leq n}$ de $[a;~b]$ et de nombres $t = (t_i)_{0 \leq i \leq n-1}$ tels que pour tout $i \in \intint{0}{n-1}$, $t_i \in [a_i;~a_{i+1}]$.

Pour toute subdivision pointée $(\sigma;~t)$ de $[a;~b]$, on construit la somme de Riemann associée à la fonction $f$ de la manière suivante:

\[
R_{\sigma,~t}(f) = \displaystyle{\sum \limits_{i \in \intint{0}{n-1}}} (a_{i+1}-a_i)f(t_i)
\]
\end{de}

On a alors le théorème suivant

\begin{theo}[Sommes de Riemann et intégration]
Une fonction $f$ est intégrable au sens de Riemann sur $[a;~b]$ si et seulement si il existe un nombre $\displaystyle{\int_a^b} f$ tel que, pour tout $\varepsilon>0$, il existe $\eta>0$ tel que, pour toute subdivision pointée $(\sigma;~t)$ de $[a;~b]$ de pas inférieur à $\eta$, on a
\[
\abs{R_{\sigma,~t}(f) - \displaystyle{\int_a^b} f} < \varepsilon
\]
\end{theo}

\begin{proof}
C'est une conséquence assez immédiate des sommes de Darboux.
\end{proof}

\subsubsection{Extension aux espaces de dimensions finies}

On suppose cette fois que $f$ est à valeurs dans un espace $E$ de dimension finie, notée $n$.

On peut alors étendre ce qui précède concernant les sommes de Riemann.

\begin{cor}[Sommes de Riemann et intégration pour un espace de dimension finie]
Une fonction $f$ est intégrable au sens de Riemann sur $[a;~b]$ si et seulement si il existe un nombre $\displaystyle{\int_a^b} f$ tel que, pour tout $\varepsilon>0$, il existe $\eta>0$ tel que, pour toute subdivision pointée $(\sigma;~t)$ de $[a;~b]$ de pas inférieur à $\eta$, on a
\[
\norm{R_{\sigma,~t}(f) - \displaystyle{\int_a^b} f} < \varepsilon
\]
\end{cor}

\begin{proof}
Soit $\left(e_i\right)_{1 \leq i \leq n}$ une base de $E$ et $f^{(i})$ les coordonnées de $f$ sur chacun des $e_i$.

Nous savons que toutes les normes de $E$ sont équivalentes. Nous choisissons la norme infinie associée à cette base.

Si $f$ est Riemann intégrable, il existe deux suites de fonctions en escaliers $\varphi_n$ et $\psi_n$ telles que
\begin{itemize}
\item[$\bullet$] pour tout $t \in [a;~b], \, \norm{\varphi_n(t)-f(t)}_{\infty} \leq \psi_n(t)$;
\item[$\bullet$] $\displaystyle{\int_a^b} \psi_n \underset{n \to +\infty}{\longrightarrow} 0$.
\end{itemize}


En considérant les coordonnées de $\varphi_n$ sur chacun des $e_i$, fonctions notées $\varphi_n^{(i)}$, on est ramené au cas réel pour chacune des coordonnées. 

Ainsi, pour tout $\varepsilon>0$ et pour tout $i$, il existe un nombre $\displaystyle{\int_a^b} f^{(i)}$ et nombre $\eta_i>0$ tel que pour toute subdivision pointée $(\sigma_i;~t_i)$ de pas inférieur à $\eta_i$, on a
\[
\abs{R_{\sigma_i,~t_i}(f^{(i)}) - \displaystyle{\int_a^b} f^{(i)}} < \varepsilon
\]

En considérant, $\eta = \min \limits_{1 \leq i \leq n} \left(\eta_i\right)$, on obtient le résultat escompté.

La réciproque se traite de la même manière, en raisonnant sur les coordonnées.
\end{proof}

\subsubsection{Formules de changement de variable}

Dans toute la suite $E$ désigne un $\R$-espace vectoriel de dimension $n$.

\begin{de}[Chemin, équivalence de chemins]
Un chemin $\Gamma$ est une application $\mathcal{C}^1$ d'un intervalle non vide $[a;~b]$ vers $E$.

On dit que deux chemins $\Gamma$ et $\tilde{\Gamma}$ définis respectivement de $[a;~b]$ et $\left [\tilde{a};~\tilde{b}\right ]$ vers $\R^n$ sont équivalents lorsqu'il existe un $\mathcal{C}^1-$difféomorphisme $\phi: [a;~b] \to \left [\tilde{a};~\tilde{b}\right ]$ tel que 
\[
\Gamma \circ \phi = \tilde{\Gamma}
\]
\end{de}


\begin{proof}
La relation d'équivalence est assez facilement prouvée: la réflexivité est évidente, la transitivité et la symétrie ne sont pas très compliquées à prouver.
\end{proof}


\begin{theo}[Intégrale d'une fonction sur un chemin]
Soit $f: E \to E$ une fonction. Soit $\Gamma$ une chemin défini sur un intervalle $[a;~b]$.

Dans le cas où la fonction $f \circ  \Gamma$ est Riemann-intégrable, on posera:
\[
\displaystyle{\int_{\Gamma}} f = \displaystyle{\int_a^b} \Gamma'(t) f \circ  \Gamma(t) \mathrm d t
\]

De plus, si $\Gamma$ et $\tilde{\Gamma}$ sont équivalents, alors $f \circ \tilde{\Gamma}$ est également Riemann-intégrable et on vérifie:
\[
\displaystyle{\int_{\Gamma}} f = \displaystyle{\int_{\tilde{\Gamma}}} f
\]
\end{theo}

\begin{proof}
On sait que le produit de deux fonctions Riemann-intégrable l'est aussi, ce qui nous prouve l'existence de $\displaystyle{\int_{\Gamma}} f$.

Reste à prouver l'égalité concernant les deux chemins équivalents. On va pour cela utiliser la caractérisation à l'aide subdivisions pointées.

On suppose donc que $\tilde{\Gamma}: [\tilde{a};~\tilde{b}] \underset{\phi}{\longrightarrow} [a;~b] \underset{\Gamma}{\longrightarrow} \C$ où $\phi$ est un $\mathcal{C}^1-$difféomorphisme.

On considère alors une subdivision pointée $(\sigma;~t)$ de $[\tilde{a};~\tilde{b}]$ et la somme de Riemann associée:
\[
R_{\sigma,~t}(f) = \displaystyle{\sum \limits_{0 \leq i \leq n-1}} \tilde{\Gamma}'\left(\tilde{t}_i\right) \times f\circ \tilde{\Gamma}\left(\tilde{t_i}\right) \times \left(\tilde{a}_{i+1}-\tilde{a}_i\right)
\]

On pose, pour la suite $a_i = \phi\left(\tilde{a}_i\right)$ et $t_i = \phi\left(\tilde{t}_i\right)$. En utilisant définition de $\tilde{\Gamma}$, ainsi que la dérivation composée on obtient:
\[
R_{\sigma,~t}(f) = \displaystyle{\sum \limits_{0 \leq i \leq n-1}} \phi'\left(\tilde{t}_i\right) \times \Gamma'(t_i) \times f\circ \Gamma(t_i) \times \left(\tilde{a}_{i+1}-\tilde{a}_i\right)
\]

Mais, d'après l'égalité des accroissements finis, on sait que, pour tout $i$, il existe $\tilde{c}_i \in \left]\tilde{a}_i;~\tilde{a}_{i+1}\right[$ tel que $\left(\phi\left(\tilde{a}_{i+1}\right)-\phi\left(\tilde{a}_i\right)\right) = \phi'\left (\tilde{c}_i\right ) \times \left (\tilde{a}_{i+1}-\tilde{a}_i\right )$. Or $\phi'$ ne s'annule pas. On en déduit:
\[
R_{\sigma,~t}(f) = \displaystyle{\sum \limits_{0 \leq i \leq n-1}} \dfrac{\phi'\left(\tilde{t}_i\right)}{\phi'\left(\tilde{c}_i\right)} \times \Gamma'(t_i) \times f\circ \Gamma(t_i) \times \left(a_{i+1}-a_i\right)
\]
On va maintenant utiliser deux arguments: d'une part l'uniforme continuité de $\phi'$ sur $\left [\tilde{a};~\tilde{b}\right ]$ et d'autre part, le fait que $\abs{\phi'}$ possède un minimum. 

On pose ainsi $m = \min \limits_{\left [\tilde{a};~\tilde{b}\right ]} \abs{\phi'}>0$ et $M$ un majorant de $\Gamma' \times f \circ \Gamma$ sur $[a;~b]$.

Pour tout $\varepsilon > 0$, il existe $\eta > 0$ tel que si la subdivision pointée $(\sigma;~t)$ est de pas inférieur à $\eta$, on a, pour tout $i$, 
\[
\abs{\phi'\left(\tilde{t}_i\right)-\phi'\left(\tilde{c}_i\right)} \leq \dfrac{m \varepsilon}{2 \times M \times (b-a)}
\]
Ce qui donne, par construction:
\[
\abs{\dfrac{\phi'\left(\tilde{t}_i\right)}{\phi'\left(\tilde{c}_i\right)}-1} \leq \dfrac{\varepsilon}{2 \times M \times (b-a)}
\]
En raison de la définition de $\displaystyle{\int_{\Gamma}} f$, il existe un nombre $\eta'>0$ éventuellement plus petit que $\eta$ tel que, si le pas de la subdivision pointée $(\sigma;~t)$ est plus petit que $\eta'$ alors le pas de l'image de cette subdivision par $\phi$ sera suffisamment petit pour que l'on ait:
\[
\norm{\displaystyle{\sum \limits_{0 \leq i \leq n-1}}  \Gamma'(t_i) \times f\circ \Gamma(t_i) \times \left(a_{i+1}-a_i\right) - \displaystyle{\int_{\Gamma}} f} \leq \dfrac{\varepsilon}{2}
\]
En effet, il suffit de constater que le pas de la subdivision image est contrôlé par le maximum de $\phi'$ en raison de l'inégalité des accroissements finis.

On obtient dans ce cas:

\begin{multline*}
\norm{\displaystyle{\sum \limits_{0 \leq i \leq n-1}} \tilde{\Gamma}'\left(\tilde{t}_i\right) \times f\circ \tilde{\Gamma}\left(\tilde{t_i}\right) \times \left(\tilde{a}_{i+1}-\tilde{a}_i\right) - \displaystyle{\int_{\Gamma}} f}  \\
\leq \norm{\displaystyle{\sum \limits_{0 \leq i \leq n-1}} \tilde{\Gamma}'\left(\tilde{t}_i\right) \times f\circ \tilde{\Gamma}\left(\tilde{t_i}\right) \times \left(\tilde{a}_{i+1}-\tilde{a}_i\right) - \displaystyle{\sum \limits_{0 \leq i \leq n-1}}   \Gamma'(t_i) \times f\circ \Gamma(t_i) \times \left(a_{i+1}-a_i\right)} \\
+ \norm{\displaystyle{\sum \limits_{0 \leq i \leq n-1}} \Gamma'(t_i) \times f\circ \Gamma(t_i) \times \left(a_{i+1}-a_i\right)- \displaystyle{\int_{\Gamma}} f}\\
\leq \norm{\displaystyle{\sum \limits_{0 \leq i \leq n-1}} \tilde{\Gamma}'\left(\tilde{t}_i\right) \times f\circ \tilde{\Gamma}\left(\tilde{t_i}\right) \times \left(\tilde{a}_{i+1}-\tilde{a}_i\right) - \displaystyle{\sum \limits_{0 \leq i \leq n-1}}   \Gamma'(t_i) \times f\circ \Gamma(t_i) \times \left(a_{i+1}-a_i\right)} + \dfrac{\varepsilon}{2}
\end{multline*}

Or, par construction, en utilisant l'inégalité triangulaire:
\begin{multline*}
\norm{\displaystyle{\sum \limits_{0 \leq i \leq n-1}} \tilde{\Gamma}'\left(\tilde{t}_i\right) \times f\circ \tilde{\Gamma}\left(\tilde{t_i}\right) \times \left(\tilde{a}_{i+1}-\tilde{a}_i\right) - \displaystyle{\sum \limits_{0 \leq i \leq n-1}}   \Gamma'(t_i) \times f\circ \Gamma(t_i) \times \left(a_{i+1}-a_i\right)} = \\
\norm{\displaystyle{\sum \limits_{0 \leq i \leq n-1}} \left( \dfrac{\phi'\left(\tilde{t}_i\right)}{\phi'\left(\tilde{c}_i\right)}-1 \right) \times   \Gamma'(t_i) \times f\circ \Gamma(t_i) \times \left(a_{i+1}-a_i\right)} \\
\leq \dfrac{\varepsilon}{2 \times M \times (b-a)} \times M \times \displaystyle{\sum \limits_{0 \leq i \leq n-1}} \abs{a_{i+1}-a_i} = \dfrac{\varepsilon}{2}
\end{multline*}

Finalement, on a bien
\[
\norm{\displaystyle{\sum \limits_{0 \leq i \leq n-1}} \tilde{\Gamma}'\left(\tilde{t}_i\right) \times f\circ \tilde{\Gamma}\left(\tilde{t_i}\right) \times \left(\tilde{a}_{i+1}-\tilde{a}_i\right) - \displaystyle{\int_{\Gamma}} f} \leq \varepsilon
\]
\end{proof}


\begin{cor}[Changement de variable]
Soit une fonction $f$ définie sur un intervalle $[a;~b]$ et à valeurs dans $E$ un espace vectoriel de dimension finie. 

On suppose que $f$ est Riemann-intégrable.

Soit $\phi$ un $\mathcal{C}^1$-difféomorphisme de $[a;~b]$ dans $\left[\tilde{a};~\tilde{b}\right]$.

Alors $\dfrac{1}{\phi'} \times f \circ \phi^{-1}$ est Riemann intégrable sur $\left[\tilde{a};~\tilde{b}\right]$ et 
\[
\displaystyle{\int_{\tilde{a}}^{\tilde{b}}} \dfrac{1}{\phi'} \times f \circ \phi^{-1} = \displaystyle{\int_{a}^{b}} f
\]
\end{cor}





